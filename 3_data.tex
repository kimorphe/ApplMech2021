%%%%%%%%%%%%%%%%%%%%%%%%%%%%%%%%%%%%%%%%%%%%%%%%%%%%%%%%%%%%%%%%%%%%%%%
計測によって得られた波形の一例を図\ref{fig:fig6}-(a)に示す.
これは石英を透過した超音波の波形を横軸を時間,縦軸を速度振幅として
示したものである.一方,図\ref{fig:fig6}-(b)は,送信探触子の
シュー先端を供試体に接触させず,空気中で自由に振動させたときの
振動速度をLDVで計測した結果で,参照波形として用いるためのものである.
これら2つの波形を比較することで,(a)に示した透過波のうち概ね13$\mu$sec
までの初動部分が直達波成分で,それ以後が板厚方向に往復を繰り返す
多重反射波成分であることが分かる.
次節では,このような直達波成分の波形を使って位相速度を求める.
参照波形と透過波形では,透過経路長(供試体の板厚)が小さいことから
到達時刻の差は1$\mu$secに満たないものである.しかしながら,
鉱物種や観測点位置によって,到達時間には変動が生じている.
図\ref{fig:fig7}-図\ref{fig:fig9}はこのことを見るために,
観測で得られた全ての波形を鉱物種毎に示したものである.これらの図は
いずれも,横軸を時間,縦軸を波形番号(観測点番号)として,速度振幅を
カラー表示したものである.
はじめに石英に対する結果である図\ref{fig:fig7}をみると,
大きな振幅をもつ初動部分の背後に,多重反射波が続く様子が
ほぼ全ての観測波形において見られる.さらに,青で示された
速度振幅が負のピークをとる時間が,細かく振動していることが分かる.
これは到達時刻が一様でない,すなわち,同じ石英でも位相速度が
観測位置によって異なることを意味する.このような速度差は
結晶異方性や結晶粒内と粒界に存在するマイクロクラックにに起因
したものと考えられる.いずれの要因によるものかはここでの計測
結果のみからは判定できないものの,局所的な超音波の伝播速度が
結晶粒スケールでの非均質性や異方性を捉えているということは言える.
以上の特徴は,図\ref{fig:fig8}や図\ref{fig:fig9}に示した
ナトリウム長石およびカリ長石にも概ね当てはまる.
一方で,これら二種類の長石に対する結果では,石英の場合に比べて
速度振幅の大きさが全体的に小さい.さらに,多重反射波の継続時間が短く,
特にカリ長石では直達波の背後に位相の揃った反射波成分がほとんど見られない.
また,振幅が負のピークを取る時刻の変動も他の鉱物種に比べて明らかに大きい.
ナトリウム長石とカリ長石は両者とも単斜晶系で,石英に比べて異方性が強い.
さらにカリ長石はへき開性があることが知られている.これらのことから
ナトリウム長石の結果とカリ長石のピーク時刻の変動が
石英に比べて大きいことは異方性の強さが,カリ長石の多重反射波
が弱いことはへき開により減衰が関係していると言える.
このように,花崗岩中の弾性波の伝播は,造岩鉱物の異方性とマイクロクラック
やへき開による影響を受け,その影響の程度は鉱物種毎に異なることが分かる.
これは同時に物理的あるいは化学的な風化によって鉱物の性状が変化した
ときの弾性波伝播への影響は,変化の程度だけでなく鉱物の組成にも依存したもの
となることを意味する。
%%弾性波から鉱物毎の特徴を逆解析できる可能性
\begin{figure}
\begin{center}
	\includegraphics[clip,scale=0.4]{Figs/Ascans.eps}
\caption{
	LDVので観測した時間波形.(a)透過波波形(石英),(b)圧電探触子シュー先端部の振動波形(参照波形).
}
\label{fig:fig5}
\end{center}
%	\vspace{-5mm}
\end{figure}
\begin{figure}
\begin{center}
	\includegraphics[clip,scale=0.5]{Figs/Bscan_Q.eps}
	\caption{石英を透過した超音波の波形.}
	\label{fig:fig6}
\end{center}
%	\vspace{-10mm}
\end{figure}
\begin{figure}
\begin{center}
	\includegraphics[clip,scale=0.5]{Figs/Bscan_N.eps}
	\caption{ナトリウム長石を透過した超音波の波形.}
	\label{fig:fig7}
\end{center}
%	\vspace{-10mm}
\end{figure}
\begin{figure}
\begin{center}
	\includegraphics[clip,scale=0.5]{Figs/Bscan_K.eps}
	\caption{カリ長石を透過した超音波の波形.}
	\label{fig:fig8}
\end{center}
%	\vspace{-10mm}
\end{figure}
