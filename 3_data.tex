%%%%%%%%%%%%%%%%%%%%%%%%%%%%%%%%%%%%%%%%%%%%%%%%%%%%%%%%%%%%%%%%%%%%%%%
計測によって得られた透過波波形の一例を図-\ref{fig:fig6}(a)に示す.
これは石英を透過した超音波の波形を,横軸を時間,縦軸を速度振幅として
示したものである.一方,図\ref{fig:fig6}-(b)は,送信探触子の
シュー先端を,供試体に接触させず空気中で自由に振動させたときの
振動速度をLDVで計測した結果で,位相速度の算出において参照波形として
用いる.これら2つの波形を比較すると,(a)に示した透過波のうち,
概ね13[$\mu$sec]までの部分が直達波成分で,それ以後の振動は
板厚方向に往復を繰り返す多重反射波によるものであると判断される.
参照波形と透過波形では,透過経路長(供試体の板厚)が短く
到達時刻の差は1$\mu$secに満たない程度である.しかしながら,
鉱物種や観測点位置によって到達時間には変動が生じている.
図-\ref{fig:fig7}$\sim$図-\ref{fig:fig9}はこのことを見るために,
全ての観測波形を鉱物種毎に示したものである.これらはいずれも
横軸を時間,縦軸を波形番号(観測点番号)とし,速度振幅を
カラー表示したものである.
はじめに石英に対する結果である図\ref{fig:fig7}をみると,
大きな振幅をもつ初動部分の背後に,多重反射波が続く様子が
ほぼ全ての観測波形で見られる.また,青で示された速度振幅が負の
ピークをとる時間は細かく振動している.これは到達時刻が一様でなく,
同じ石英でも位相速度が観測位置によって異なることを意味する.
このような到達時間差は,結晶異方性やマイクロクラックに起因
したもので,局所的な超音波の伝播速度を見ることで,結晶粒スケール
での非均質性や異方性を捉えることができることを示している.
以上の特徴は図-\ref{fig:fig8}と図-\ref{fig:fig9}に示した
ナトリウム長石およびカリ長石にも概ね当てはまる.
一方で,これら二種類の長石に対する結果では,石英の場合に比べて
速度振幅の大きさが全体的に小さい.さらに,多重反射波の継続時間が短く,
特にカリ長石では直達波の背後に位相の揃った反射波成分がほとんど見られない.
また,振幅が負のピークを取る時刻の変動もカリ長石では他の鉱物種に比べて大きい.
ナトリウム長石とカリ長石は両者とも単斜晶系で,石英に比べて結晶構造の対称性が低い.
また,万成花崗岩のカリ長石にはへき開が認められる.
これらのことから,2種類の長石でピーク時刻の変動が石英に比べて大きいことには
異方性のタイプが関係し,カリ長石の多重反射波が弱いことはへき開による減衰が関係
していることが示唆される.
%このように,花崗岩中の弾性波は造岩鉱物の異方性やへき開による影響を受け,
%鉱物種毎に異なる挙動を示す.このことは,例えば風化によって鉱物の性状が変化しとき,
%弾性波伝播への影響が,風化の程度だけでなく鉱物組成にも依存したものとなる
%ことが期待でき,工学的に有用な変化がとなることを意味する.
%%弾性波から鉱物毎の特徴を逆解析できる可能性
\begin{figure}
\begin{center}
	\includegraphics[clip,scale=0.4]{Figs/Ascans.eps}
\caption{
	LDVので観測した時刻歴波形.(a)透過波波形(石英),(b)圧電探触子シュー先端部の振動波形(参照波形).
}
\label{fig:fig5}
\end{center}
%	\vspace{-5mm}
\end{figure}
\begin{figure}
\begin{center}
	\includegraphics[clip,scale=0.5]{Figs/Bscan_Q.eps}
	\caption{石英を透過した超音波の波形.}
	\label{fig:fig6}
\end{center}
%	\vspace{-10mm}
\end{figure}
\begin{figure}
\begin{center}
	\includegraphics[clip,scale=0.5]{Figs/Bscan_N.eps}
	\caption{ナトリウム長石を透過した超音波の波形.}
	\label{fig:fig7}
\end{center}
%	\vspace{-10mm}
\end{figure}
\begin{figure}
\begin{center}
	\includegraphics[clip,scale=0.5]{Figs/Bscan_K.eps}
	\caption{カリ長石を透過した超音波の波形.}
	\label{fig:fig8}
\end{center}
%	\vspace{-10mm}
\end{figure}
