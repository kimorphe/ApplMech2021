ここで,実験で得られた位相速度$c_p^{(\alpha)}$と単結晶の位相速度$v$の比較を行う.
単結晶の位相速度の計算には,文献に与えられた弾性係数$C_{ijkl}$と密度$\rho$
を用い,Christoffelの方程式を解くことによって求める.
Christoffelの方程式は,平面波の振動および進行方向を表す単位ベクトルを
それぞれ
\begin{equation}
	\fat{d}=(d_1,\, d_2,\, d_3), \ \ 
	\fat{p}=(p_1,\, p_2,\, p_3)
	\label{eqn:unit_vecs}
\end{equation}
とすれば、次のように表すことができる.
\begin{equation}
	\left( C_{ijkl}p_j p_k - \rho v^2 \right)d_k=0
	\label{eqn:Christoffel}
\end{equation}
式(\ref{eqn:Christoffel})を与えられた進行方向ベクトル$\fat{p}$
に対して解くことで,$\fat{p}$方向への疑似縦波と疑似横波の
位相速度$v$を求めることができる.ここでは,$\fat{p}$を
\begin{equation}
	\fat{p}=\left(
		\sin\theta \cos\phi, \,
		\sin\theta \sin\phi, \,
		\cos\theta
	\right)
	\label{eqn:p_polar}
\end{equation}
球座標表示し,
\begin{equation}
	0 \leq \theta \leq \pi, \ \ 
	0 \leq \phi \leq 2\pi
	\label{eqn:}
\end{equation}
の範囲において,$v(\theta, \phi)$を1度刻みで求めた.
図\ref{fig:***}は,その結果を立体角$\Delta \Omega = \sin\theta \Delta \theta \Delta phi$
で重み付けした頻度分布として示したものである.これらは,多数の単結晶が一様な方位分布をもつと
仮定したときの,縦波位相速度の分布を表す.
実験で求めた鉱物粒の位相速度$c^{(\alpha)}_p$と単結晶の位相速度分布を比較すると,
単結晶の位相速度は花崗岩の鉱物粒に対する結果よりも平均的に大きな値を取ることが分かる。
これは,花崗岩では鉱物粒に含まれるマイクロクラックによって軟化し,理想的な単結晶よりも
位相速度が低下するためと解釈することができる.
位相速度の分布幅についてみれば,鉱物種間での大小関係は単結晶と花崗岩とも
カリ長石>ナトリウム長石>石英の順で変わりはない.
ただし,分布形状は単結晶と花崗岩の場合で大きく異なっている.
例えば,石英の場合,花崗岩に対する結果では分布幅は狭く,単結晶に比べて異方性が弱められている.
このことは,石英中のマイクロクラックには方向性が弱く,その結果単結晶の場合よりも等方的に振る舞う
ことを示唆している。一方、カリ長石とナトリウム長石は単結晶と比べて位相速度が低下する点は
石英と共通するが,分布幅は石英に比べて大きくクラックの存在下でも異方性を示す。
特にカリ長石の位相速度は分布幅が広く,これは、へき開の存在により著しい軟化が起きるものの,
へき開の方向性のために強い異方性が残ると考えれば解釈がつく.
