本節では,縦波位相速度$c_p$の計算方法をはじめに述べる.
次に,鉱物種ごとに求めた位相速度の平均と分布を求め,
鉱物種による位相速度の違いと速度分散の有無について明らかにする.
\subsection{位相速度の計算方法}
位相速度$c_p$を,図\ref{fig:fig5}-(b)に示した参照波形と個々の透過波形の
位相差から次の手順で求める.以下では,計測した透過波の時間波形を$a_{raw}(t)$,
参照波形を$a_{ref}(t)$と表す.
\begin{enumerate}
\item
直達波成分の切り出し: 
位相速度の計算には直達波を用いるため,観測波形$a_{raw}(t)$に窓関数$W(t;T_w)$を作用させ
\begin{equation}
	a(t):=W(t-t_b;\tau)a_{raw}(t)
\end{equation}
とすることで,直達波$a(t)$を切り出す.ここで,$t_b$は時間軸上における窓関数の位置を,
$\tau$は窓関数の幅を決定するパラメータを表す.本研究では,$W(t;\tau)$として
次の式で与えられるHann Windowを用いる.
\begin{equation}
	W(t;\tau):=\left\{
	\begin{array}{cl}
		\frac{1}{2}\left[ 1-\cos \left\{ \pi\left(1+t/\tau \right) \right\} \right],
		 & \left| t \right| < \tau \\
		0, & {\rm otherwise}
	\end{array}
	\right.
\end{equation}
Hann Windowのパラメータ$\tau$は$1MHz$以上の周波数成分を捉えることができるよう
$\tau=0.6$[$\mu$sec]とし,位置$t_b$は直達波の位置を振幅の正負が切り替わる
時刻として見積もって与えるようにした.
なお,Hann Windowを選択した理由は,少数のパラメータを設定するだけでよく使い易いこと,
有限な台をもつため,多重反射波の成分を完全に切り落とすことができることの二点である.
\item
フーリエ変換と逆畳み込み:
時間に関するフーリエ変換を,角周波数を$\omega$として
	\begin{equation}
		F(\omega) ={\cal F} \left\{ f(t)\right\}:=\int f(t) e^{i\omega t}dt 
		\label{eqn:Fourier}
	\end{equation}
で定義し,$a(t)$と$a_{ref}(t)$のフーリエ変換
	\begin{equation}
		A(\omega):={\cal F} \left\{ a(t) \right\}, \ \ 
		A_{ref}(\omega):={\cal F} \left\{ a_{ref} (t)\right\}
	\end{equation}
をFFTでそれぞれ求める.
これらの逆畳み込みをとることで,信号$a(t)$と$a_{ref}(t)$の位相差$\Delta \phi(\omega)$
を次の式から求める.
\begin{equation}
	\frac{A(\omega)}{A_{ref}(\omega)}
	=
	\left|
	\frac{A(\omega)}{A_{ref}(\omega)}
	\right|
	e^{-i\Delta \phi(\omega)}
	\label{eqn:deconv}
\end{equation}
\item
位相差のアンラップ:
	位相差$\Delta \phi(\omega)$を低周波側から高周波側に向かってアンラップする.
	このとき,周波数$0$の近傍ではノイズの影響が大きく誤った位相の値を持つ可能性が高いため,
	参照波の周波数帯域下限付近である500[kHz]を起点としてアンラップを行う.
	なお,周波数500[kHz]であれば,今回の実験条件において縦波の波長が供試体の
	板厚(透過距離)を下回ることは無く,位相の不定性は問題にならない.
\item
位相速度の計算:
	アンラップされた位相差$\Delta \phi(\omega)$と,供試体の厚さ$h$(透過距離)を用い,
\begin{equation}
	c_p(\omega)=\frac{\omega h}{\Delta \phi(\omega)}
	\label{eqn:cp_phi}
\end{equation}
	で位相速度を周波数毎に計算する.
\end{enumerate}
以上の手順で求めた位相速度と,対応する鉱物種を明示する場合には,
位相速度を$c_p^{(\alpha)}$と書くことにする.ここに,$\alpha$はQt, Na, Kのいずれかで,
それぞれ,石英,ナトリウム長石,カリ長石を意味する.また,位相速度の値は計測波形の数$N_\alpha$
だけ求められる.そこで,第$m$番目の計測波形に対して得られた位相速度を指す場合は
$c_{p,m}^{(\alpha)}$と表記し,鉱物種$\alpha$に対する波形データについて
得られた位相速度の全体を指す場合には$\left\{ c_{p,m}^{(\alpha)}\right\}$と書く.
さらに,鉱物種$\alpha$に対する位相速度の平均を
\begin{equation}
	\left< c_p^{(\alpha)}\right>:= \frac{1}{N_\alpha}\sum_{m}^{N_\alpha} c_{p,m}^{\alpha }
	\label{eqn:}
\end{equation}
と表す.以下では,この意味での位相速度の平均や分布について議論する.
\subsection{位相速度の平均}
鉱物種毎に求めた位相速度の平均と周波数の関係を図\ref{fig:figfig9}に示す.
この図は,周波数に対して$\left< c_p^{(Q)}\right>, 
\left< c_p^{(Na)}\right>$,および $\left< c_p^{(K)}\right>$をプロット
したもので.平均的な速度分散関係を示したものである.
この結果から明らかなように,鉱物種によらず位相速度はほぼ一定値となっており,
ここに示した周波数帯域ではほとんど速度分散は見られないことが分かる.
また,位相速度の大きさは
\begin{equation}
	\left< c_p^{(Q)} \right> 
	< 
	\left< c_p^{(K)} \right> 
	< 
	\left< c_p^{(Na)}\right>
	\label{eqn:}
\end{equation}
で周波数によって大小関係が逆転することは無い.
ただし,いずれの鉱物種でも位相速度は概ね5.4[km/s]
で,鉱物種間での差も約$\pm$ 0.2[km/s],すなわち3.7$\%$程度と
平均値を見る限りあまり大きくない.
\begin{figure}
\begin{center}
	\includegraphics[clip,scale=0.5]{Figs/cp_mean_f.eps}
	\caption{位相速度の平均値と周波数の関係.}
	\label{fig:fig9}
\end{center}
	\vspace{-10mm}
\end{figure}
\begin{figure}
\begin{center}
	\includegraphics[clip,scale=0.5]{Figs/hist_Q.eps}
	\caption{位相速度の分布(石英).}
	\label{fig:fig10}
\end{center}
	\vspace{-10mm}
\end{figure}
\begin{figure}
\begin{center}
	\includegraphics[clip,scale=0.5]{Figs/hist_N.eps}
	\caption{位相速度の分布(ナトリウム長石).}
	\label{fig:fig11}
\end{center}
	\vspace{-10mm}
\end{figure}
\begin{figure}
\begin{center}
	\includegraphics[clip,scale=0.5]{Figs/hist_K.eps}
	\caption{位相速度の分布(カリ長石).}
	\label{fig:fig12}
\end{center}
	\vspace{-10mm}
\end{figure}
%%%%%%%%%%%%%%%%%%%%%%%%%%
