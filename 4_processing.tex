本節では,縦波位相速度$c_p$の計算方法をはじめに述べる.
次に,鉱物種ごとに求めた位相速度の平均と頻度分布を求め,
鉱物種による位相速度の違いと速度分散の有無について明らかにする.
\subsection{位相速度の計算方法}
位相速度$c_p$は,図-\ref{fig:fig5}(b)に示した参照波形と個々の透過波形の
位相差から次の手順に従って求める.以下では,透過波の時間波形を$a_{raw}(t)$,
参照波形を$a_{ref}(t)$と表す.
\begin{enumerate}
\item
直達波成分の切り出し: 
位相速度の計算には直達波を用いる.そこで,観測波形$a_{raw}(t)$に窓関数$W(t;T_w)$を作用させ
\begin{equation}
	a(t):=W(t-t_b;\tau)a_{raw}(t)
\end{equation}
とすることで,直達波$a(t)$を切り出す.ここで,$t_b$は時間軸上における窓関数位置を,
$\tau$は窓関数の幅を決定するパラメータを表す.本研究では,$W(t;\tau)$に
次式で与えられるHann Windowを用いる.
\begin{equation}
	W(t;\tau):=\left\{
	\begin{array}{cl}
		\frac{1}{2}\left[ 1-\cos \left\{ \pi\left(1+t/\tau \right) \right\} \right],
		 & \left| t \right| < \tau \\
		0, & {\rm otherwise}
	\end{array}
	\right.
\end{equation}
Hann Windowのパラメータ$\tau$は1MHz以上の周波数成分を捉えることができるよう
$\tau=0.6$[$\mu$sec]とする.一方,位置$t_b$は直達波の振幅の正負が切り替わる
時刻を波形毎に与えるようにした.なお,Hann Windowを選択した理由は,有限な台をもつために
多重反射波の成分を完全に切り落とすことができることによる.
\item
フーリエ変換と逆畳み込み:
角周波数を$\omega$とし,時間に関するフーリエ変換を
	\begin{equation}
		F(\omega) ={\cal F} \left\{ f(t)\right\}:=\int f(t) e^{i\omega t}dt 
		\label{eqn:Fourier}
	\end{equation}
で定義する.$a(t)$と$a_{ref}(t)$のフーリエ変換
	\begin{equation}
		A(\omega):={\cal F} \left\{ a(t) \right\}, \ \ 
		A_{ref}(\omega):={\cal F} \left\{ a_{ref} (t)\right\}
	\end{equation}
をFFTでそれぞれ求め,これらの逆畳み込みにより,信号$a(t)$と$a_{ref}(t)$の位相差$\Delta \phi(\omega)$
を次の式から求める.
\begin{equation}
	\frac{A(\omega)}{A_{ref}(\omega)}
	=
	\left|
	\frac{A(\omega)}{A_{ref}(\omega)}
	\right|
	e^{-i\Delta \phi(\omega)}
	\label{eqn:deconv}
\end{equation}
\item
位相差のアンラップ:
	位相差$\Delta \phi(\omega)$を低周波側から高周波側に向かってアンラップする.
	このとき,周波数$f=0$の近傍では,ノイズが大きく誤った位相の値が得られる可能性が高い.
	そこで,参照波の周波数帯域下限付近である$f=$500[kHz]を起点としてアンラップを行う.
	なお,$f=$500[kHz]であれば,今回の実験条件において縦波の波長が供試体の板厚(透過距離)を
	下回ることは無く,位相の不定性は生じない.
\item
位相速度の計算:
	アンラップされた位相差$\Delta \phi(\omega)$と,供試体厚さ$h$(透過距離)を用い,
	\begin{equation}
		c_p(\omega)=\frac{\omega h}{\Delta \phi(\omega)}
		\label{eqn:cp_phi}
	\end{equation}
	で周波数毎に位相速度を計算する.
\end{enumerate}
以上の手順で求めた位相速度と鉱物種の対応を明示する場合,位相速度を$c_p^{(\alpha)}$と書くことにする.
ここに,$\alpha$はQt,Na,Kのいずれかとし,それぞれ,石英,ナトリウム長石,カリ長石を表す.
位相速度$c_p^{(\alpha)}$は計測波形の数$N_\alpha$だけ求められる.そこで,鉱物種$\alpha$の
第$m$番目の計測波形に対する位相速度を$c_{p,m}^{(\alpha)}$と表記し,鉱物種$\alpha$に対する位相速度
データの全体を$\left\{ c_{p,m}^{(\alpha)}\right\}$と書く.
さらに,位相速度の平均を
\begin{equation}
	\left< c_p^{(\alpha)}\right>:= \frac{1}{N_\alpha}\sum_{m}^{N_\alpha} c_{p,m}^{(\alpha)}
	\label{eqn:}
\end{equation}
と表す.以下ではこの意味での位相速度の平均や$\left\{ c_{p,m}^{(\alpha)}\right\}$の
頻度分布について調べる.
%
%
\subsection{位相速度の平均}
鉱物種毎に求めた位相速度の平均と周波数の関係を図-\ref{fig:fig9}に示す.
この図は周波数$f$[MHz]に対して
$\left< c_p^{(\alpha)}\right>, (\alpha=$Qt,Na,K)をプロットしたものである.
このグラフから明らかなように,位相速度は鉱物種によらずほぼ一定値となっており,
ここに示した周波数帯域ではほとんど速度分散は無いことが分かる.
また,平均位相速度の大小は
\begin{equation}
	\left< c_p^{(Qt)} \right> 
	< 
	\left< c_p^{(K)} \right> 
	< 
	\left< c_p^{(Na)}\right>
	\label{eqn:}
\end{equation}
で一貫しており,周波数に応じて大小が逆転することもない.
位相速度の大きさはいずれの鉱物種でも概ね5.4[km/s]で,これは
岩石コアで計測したマクロな縦波速度に近い値となっている.
なお,鉱物種間での差は平均位相速度の差は約$\pm$ 0.2[km/s],
位相速度との比率でみたと場合にも3.7$\%$程度とあまり大きなものではない.
\begin{figure}
\begin{center}
	\includegraphics[clip,scale=0.5]{Figs/cp_mean_f.eps}
	\caption{位相速度の平均値と周波数の関係.}
	\label{fig:fig9}
\end{center}
	\vspace{-10mm}
\end{figure}
%
%
\subsection{位相速度の頻度分布}
$\left\{ c^{(\alpha)}_{p,m}\right\}$の頻度分布を図-\ref{fig:fig10}$\sim$図-\ref{fig:fig12}に示す.
これらは横軸を周波数,縦軸を位相速度$c^{(\alpha)}_{p,m}$にとり,頻度をカラー表示したものである.
白の実線は図\ref{fig:fig9}に示した平均位相速度$\left<c^{(\alpha)}_p\right>$
と周波数の関係を再掲したものである.これらの結果を見ると,位相速度の頻度分布は周波数によって
大きく変化しないことが分かる.一方,頻度分布の形状は鉱物種によって大きく異なっている.
図-\ref{fig:fig10}に示した石英に対する結果では位相速度の分布は平均値周辺に集中している.
これに対して,ナトリウム長石とカリ長石では位相速度は広い範囲に分布しており複数のピークが現れている.
特に,カリ長石の場合,頻度分布は上下に非対称で4$\sim$7[km/s]に渡る非常に広い分布となっている.
このような頻度分布を正規化すれば,位相速度が従う確率密度関数を得ることができる.
勿論,正規化により分布形状や平均,分散が変化することは無く,頻度分布と確率密度関数
のいずれとして分布を表示するかに実質的な違いはほとんどない.ただし,頻度分布による表示には,
どの程度のサンプル数から構成された分布であるかを読み取ることができるという利点があるため,
以下では分布の正規化は行わず,頻度分布をそのまま示すことにする.


鉱物種による分布の差をより明確に示すために,
周波数$f=\frac{\omega}{2\pi}$=1.0$\sim$2.0[MHz]の全ての周波数で
得られた位相速度
\[
	\left\{ c_{p,m}^{(\alpha)}(\omega)\left| m=0,\dots N_\alpha ,\  f\in \left[ 1.0,2.0\right][{\rm MHz}] \right.\right\}
\]
に対する頻度分布を図-\ref{fig:fig13}に示す.
この図にあるように,石英とナトリウム長石は概ね対称な分布であるのに対し,
カリ長石は非対称で分布の幅が非常に広い.実際,これらの3つの頻度分布について
平均と標準偏差を求めると表\ref{tbl:tbl1}のようになる.
\begin{table}[htb]
	\caption{縦波位相速度の平均と標準偏差[km/s]}
  \begin{tabular}{c||c|c|c|c}
	  鉱物種 & 石英 & Na長石 & カリ長石 & 全データ\\
	  \hline
	  平均 & 5.37 & 5.53 & 5.41 & 5.43  \\
	  \hline
	  標準偏差 & 0.44 & 0.56 & 0.90 & 0.70  \\
  \end{tabular}
\label{tbl:tbl1}
\end{table}
なお,表\ref{tbl:tbl1}において全データとは,3つの鉱物種について得られた
全ての位相速度に対する平均と標準偏差を意味する.\\

ここで,実験で得られた位相速度$c_p^{(\alpha)}$と文献値\cite{AGU}から求めた
鉱物単結晶の位相速度$v$の比較を行う.
単結晶鉱物の位相速度は文献に与えられた弾性係数$C_{ijkl}$と
密度$\rho$を用い,Christoffelの方程式を解くことによって求める.
Christoffelの方程式は,平面波の振動および進行方向を表す単位ベクトル
をそれぞれ
\begin{equation}
	\fat{d}=(d_1,\, d_2,\, d_3), \ \ 
	\fat{p}=(p_1,\, p_2,\, p_3)
	\label{eqn:unit_vecs}
\end{equation}
とすれば,次のように表すことができる.
\begin{equation}
	\left( C_{ijkl}p_j p_l - \rho v^2 \delta_{ik} \right)d_k=0
	\label{eqn:Christoffel}
\end{equation}
ここでインデックスの範囲は1,2,3とし,$\delta_{ik}$はクロネッカーデルタを
表す.式(\ref{eqn:Christoffel})を与えられた進行方向ベクトル$\fat{p}$
に対して$v$について解くことで,$\fat{p}$方向への疑似縦波と疑似横波の
位相速度が求められる.ここでは$\fat{p}$を
\begin{equation}
	\fat{p}=\left(
		\sin\theta \cos\phi, \,
		\sin\theta \sin\phi, \,
		\cos\theta
	\right)
	\label{eqn:p_polar}
\end{equation}
と球座標表示し,
\begin{equation}
	0 \leq \theta \leq \pi, \ \ 
	0 \leq \phi \leq 2\pi
	\label{eqn:}
\end{equation}
の範囲で,$v(\theta, \phi)$を刻み幅$\Delta \theta=\Delta \phi =$1度で求めた.
図-\ref{fig:fig14}はその結果を立体角$\Delta \Omega = \sin\theta \Delta \theta \Delta \phi$
で重み付けした頻度分布として示したものである.
これら,多数の単結晶が一様な方位分布をもつと仮定したときに,疑似縦波の位相速度
が従う確率(頻度)分布を表す.
実験で求めた鉱物粒の位相速度$c^{(\alpha)}_p$と単結晶の位相速度分布を比較すると,
単結晶の位相速度は花崗岩の鉱物粒に対する結果よりも平均的に大きな値を取ることが分かる.
これは,花崗岩の鉱物粒は粒界や粒内のマイクロクラックによって軟化し,
理想的な単結晶よりも弾性係数や位相速度が低下するためと解釈できる.
一方,位相速度の分布幅についてみれば,鉱物種間での大小関係は単結晶と花崗岩の場合とも,
カリ長石$>$ナトリウム長石$.$石英の順で変わりはない.ただし,分布形状は単結晶と花崗岩の
場合で大きく異なっている.例えば,石英の場合,花崗岩に対する結果では分布幅が狭い.
これは,石英粒が花崗岩内で非常に強く配向しているか,粒内のき裂によって異方性が単結晶の
場合よりも弱められていることを意味する.また,ナトリウム長石の場合,単結晶と比べて
位相速度が低下する点は同じだが,分布幅は石英程,単結晶との比較で小さく
なっていない.これに対してカリ長石は,分布幅は単結晶場合よりも一層広く,
へき開の存在による著しい軟化が起きるものの,花崗岩中でも強い異方性を示すことが分かる.
以上の考察は,カリ長石やナトリウム長石の粒内や粒界では,石英粒間よりも強い散乱源
になることをことを意味し,花崗岩中を伝播した弾性波の散乱減衰のメカニズムを理解する
上での有用な手がかりになると考えられる.
\begin{figure}
\begin{center}
	\includegraphics[clip,scale=0.5]{Figs/hist_Q.eps}
	\caption{位相速度の分布(石英).}
	\label{fig:fig10}
\end{center}
	\vspace{-10mm}
\end{figure}
\begin{figure}
\begin{center}
	\includegraphics[clip,scale=0.5]{Figs/hist_N.eps}
	\caption{位相速度の分布(ナトリウム長石).}
	\label{fig:fig11}
\end{center}
	\vspace{-10mm}
\end{figure}
\begin{figure}
\begin{center}
	\includegraphics[clip,scale=0.5]{Figs/hist_K.eps}
	\caption{位相速度の分布(カリ長石).}
	\label{fig:fig12}
\end{center}
	\vspace{-10mm}
\end{figure}
\begin{figure}
\begin{center}
	\includegraphics[clip,scale=0.5]{Figs/hist_all.eps}
	\caption{位相速度の頻度分布(全データを使用).}
	\label{fig:fig13}
\end{center}
	\vspace{-10mm}
\end{figure}
\begin{figure}
\begin{center}
	\includegraphics[clip,scale=0.5]{Figs/monocrystal.eps}
	\caption{位相速度の頻度分布(単結晶の場合).}
	\label{fig:fig14}
\end{center}
	\vspace{-10mm}
\end{figure}
%%%%%%%%%%%%%%%%%%%%%%%%%%
