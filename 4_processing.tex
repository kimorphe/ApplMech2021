本節では,縦波位相速度$c_p$の計算方法をはじめに述べる.
次に,鉱物種ごとに求めた位相速度の平均と分布を求め,
鉱物種による位相速度の違いと速度分散の有無について明らかにする.
\subsection{位相速度の計算方法}
位相速度$c_p$を,図\ref{fig:fig5}-(b)に示した参照波形と個々の透過波形の
位相差から次の手順で求める.以下では,計測した透過波の時間波形を$a_{raw}(t)$,
参照波形を$a_{ref}(t)$と表す.
\begin{enumerate}
\item
直達波成分の切り出し: 
位相速度の計算には直達波を用いるため,観測波形$a_{raw}(t)$に窓関数$W(t;T_w)$を作用させ
\begin{equation}
	a(t):=W(t-t_b;\tau)a_{raw}(t)
\end{equation}
とすることで,直達波$a(t)$を切り出す.ここで,$t_b$は時間軸上における窓関数の位置を,
$\tau$は窓関数の幅を決定するパラメータを表す.本研究では,$W(t;\tau)$として
次の式で与えられるHann Windowを用いる.
\begin{equation}
	W(t;\tau):=\left\{
	\begin{array}{cl}
		\frac{1}{2}\left[ 1-\cos \left\{ \pi\left(1+t/\tau \right) \right\} \right],
		 & \left| t \right| < \tau \\
		0, & {\rm otherwise}
	\end{array}
	\right.
\end{equation}
Hann Windowのパラメータ$\tau$は$1MHz$以上の周波数成分を捉えることができるよう
$\tau=0.6$[$\mu$sec]とし,位置$t_b$は直達波の位置を振幅の正負が切り替わる
時刻として見積もって与えるようにした.
なお,Hann Windowを選択した理由は,少数のパラメータを設定するだけでよく使い易いこと,
有限な台をもつため,多重反射波の成分を完全に切り落とすことができることの二点である.
\item
フーリエ変換と逆畳み込み:
時間に関するフーリエ変換を,角周波数を$\omega$として
	\begin{equation}
		F(\omega) ={\cal F} \left\{ f(t)\right\}:=\int f(t) e^{i\omega t}dt 
		\label{eqn:Fourier}
	\end{equation}
で定義し,$a(t)$と$a_{ref}(t)$のフーリエ変換
	\begin{equation}
		A(\omega):={\cal F} \left\{ a(t) \right\}, \ \ 
		A_{ref}(\omega):={\cal F} \left\{ a_{ref} (t)\right\}
	\end{equation}
をFFTでそれぞれ求める.
これらの逆畳み込みをとることで,信号$a(t)$と$a_{ref}(t)$の位相差$\Delta \phi(\omega)$
を次の式から求める.
\begin{equation}
	\frac{A(\omega)}{A_{ref}(\omega)}
	=
	\left|
	\frac{A(\omega)}{A_{ref}(\omega)}
	\right|
	e^{-i\Delta \phi(\omega)}
	\label{eqn:deconv}
\end{equation}
\item
位相差のアンラップ:
	位相差$\Delta \phi(\omega)$を低周波側から高周波側に向かってアンラップする.
	このとき,周波数$0$の近傍ではノイズの影響が大きく誤った位相の値を持つ可能性が高いため,
	参照波の周波数帯域下限付近である500[kHz]を起点としてアンラップを行う.
	なお,周波数500[kHz]であれば,今回の実験条件において縦波の波長が供試体の
	板厚(透過距離)を下回ることは無く,位相の不定性は問題にならない.
\item
位相速度の計算:
	アンラップされた位相差$\Delta \phi(\omega)$と,供試体の厚さ$h$(透過距離)を用い,
\begin{equation}
	c_p(\omega)=\frac{\omega h}{\Delta \phi(\omega)}
	\label{eqn:cp_phi}
\end{equation}
	で位相速度を周波数毎に計算する.
\end{enumerate}
以上の手順で求めた位相速度と,対応する鉱物種を明示する場合には,
位相速度を$c_p^{(\alpha)}$と書くことにする.ここに,$\alpha$は$Qt,Na,K$のいずれかで,
それぞれ,石英,ナトリウム長石,カリ長石を意味する.また,位相速度の値は計測波形の数$N_\alpha$
だけ求められる.そこで,第$m$番目の計測波形に対して得られた位相速度を指す場合は
$c_{p,m}^{(\alpha)}$と表記し,鉱物種$\alpha$に対する波形データについて
得られた位相速度の全体を指す場合には$\left\{ c_{p,m}^{(\alpha)}\right\}$と書く.
さらに,鉱物種$\alpha$に対する位相速度の平均を
\begin{equation}
	\left< c_p^{(\alpha)}\right>:= \frac{1}{N_\alpha}\sum_{m}^{N_\alpha} c_{p,m}^{(\alpha)}
	\label{eqn:}
\end{equation}
と表す.以下ではこの意味での位相速度の平均や分布について議論する.
%
%
\subsection{位相速度の平均}
鉱物種毎に求めた位相速度の平均と周波数の関係を図\ref{fig:fig9}に示す.
この図は周波数に対して$\left< c_p^{(Qt)}\right>$と$\left< c_p^{(Na)}\right>$,
および $\left< c_p^{(K)}\right>$をプロットしたものである.
この結果から明らかなように,鉱物種によらず位相速度はほぼ一定値となっており,
ここに示した周波数帯域ではほとんど速度分散を示さないことが分かる.
また,位相速度の大小は
\begin{equation}
	\left< c_p^{(Qt)} \right> 
	< 
	\left< c_p^{(K)} \right> 
	< 
	\left< c_p^{(Na)}\right>
	\label{eqn:}
\end{equation}
で,この関係は一貫しており周波数に応じて逆転する部分も無い.
ただし,いずれの鉱物種でも位相速度は概ね5.4[km/s]と
岩石コアで計測した際の縦波速度に近い値となっており,
鉱物種間での差は約$\pm$ 0.2[km/s],位相速度との比で言えば3.7$\%$程度と
平均値を見る限りあまり大きくない.
\begin{figure}
\begin{center}
	\includegraphics[clip,scale=0.5]{Figs/cp_mean_f.eps}
	\caption{位相速度の平均値と周波数の関係.}
	\label{fig:fig9}
\end{center}
	\vspace{-10mm}
\end{figure}
%
%
\subsection{位相速度の頻度分布}
$\left\{ c^{(\alpha)}_{p,m}\right\}$の頻度分布を図\ref{fig:fig10}$\sim$図\ref{fig:fig12}に示す.
これらの図は横軸を周波数,縦軸は位相速度$c^{(\alpha)}_{p,mn}$をとり,
頻度をカラー表示したもので,白の実線は図\ref{fig:fig9}に示した平均位相速度$\left<c^{(\alpha)}_p\right>$
と周波数の関係を再掲したものである.これらの結果を見ると,位相速度の頻度分布は周波数によって
大きく変化しないことが分かる。これに対して,鉱物種によって頻度分布は大きく異なっている.
図\ref{fig:fig10}に示した石英に対する結果では、位相速度の分布は平均値周辺に集中している.
一方,ナトリウム長石とカリ長石では位相速度は広い範囲に分布しており、複数のピークがある。
特にカリ長石の場合、頻度分布は上下に非対称で4$\sim$7[km/s]に渡る非常に広い分布となっている。
鉱物種による分布の差をより明確に示すために,周波数$f$=1.0$\sim$2.0[MHz]の全ての周波数で
得られた位相速度
\[
	\left\{ c_{p,m}^{(\alpha)}(\omega)\left| m=0,\dots N_\alpha ,\ 1.0\leq f \leq 2.0[{\rm MHz}], \right.\right\}
\]
について計算した頻度分布を図\ref{fig:fig13}に示す.
この結果では、石英、ナトリウム長石は平均値に対して概ね対称な分布で、
カリ長石だけが非対称な幅の広い分布になっている。また、
ナトリウム長石とカリ長石は多峰的であることが、石英と異なる点の一つ。
次に、比較を行うため、文献値を用いて計算した位相速度の頻度分布を
図\ref{fig:fig14}に示す。
この結果は、文献で与えられた弾性係数を用い、Christoffel方程式を
\begin{equation}
	\left( C_{ijkl}n_kn_l-\rho c^2\delta _{ij}\right)d_j =0
\end{equation}
解いて求めた縦波位相速度から作成したものである。ただし,
$C_{ijkl}$は弾性係数テンソル,$\fat{n}=(n_1,n_2,n_3)$は伝播方向を
表す単位ベクトル,$\fat{d}=(d_1,d_2,d_3)$は偏光方向ベクトル,$\rho$は密度,$c$は位相速度を表す.
これは、欠陥の無い理想的単結晶の位相速度を求めることを意味する。
この方法でそれぞれの鉱物種について位相速度を
\[
	\fat{n}=(n_1,n_2,n_3)=(\sin\theta\cos\phi,\, \sin\theta\cos\phi, \cos\theta)
\]
として,$\theta,\phi$を1度刻みで変化させて、全ての伝播方向について求め、
その結果を立体角を$\sin\theta$を重みとして作成したヒストグラムが図\ref{fig:fig14}である。
これは、結晶主軸の方向に配向が全く無い場合の結果を意味している。
以下ではこれを理想的単結晶の位相速度分布と呼ぶ。
理想的単結晶の位相速度分布の特徴は、多峰的であること、平均値が今回の実験で得られた
値よりも大きく6〜7[km/s]となることにある。
従って、花崗岩供試体で得られた弾性波速度が、理想的単結晶のものより小さい理由は
、マイクロクラックやへき開の存在によると考えられる。
実験では結晶単位での計測が行われている訳ではなく、周囲の鉱物粒の影響や粒界の
影響を受ける。しかし、3種類の造岩鉱物全ての音速が、理想的単結晶のものよりも
小さくなることは、空間的な解像度の問題でない。
特にカリ長石は結晶サイズが大きく、き裂やへき開が存在しない場合、単結晶に近い
値が得られると考えられる。
なお、多峰的であることは、結晶軸の方向に偏りがある(配向)があるということではなく、
結晶の方向は完全にランダムだが、広い方向で弾性波速度が近い値をとるということを意味しているl
例えば、等方性体では、位相速度は方向に依らないため、その頻度分布を描くと
デルタ関数になる。すなわち、頻度分布のピークは、同じ弾性波速度をとる伝播方向が
多数存在することを意味している。このことを踏まえると、石英は
比較的等方性体に近い振る舞いをしていることが分かる。
分布幅の広い、ナトリウム長石とカリ長石はより強い音響異方性を示している。
また、頻度のピークが理想的単結晶よりも著しく低い側にシフトするのは、
マイクロクラックの存在によって結晶が軟化するためと考えられる。
その結果、石英は音響異方性が弱められるが、ナトリウム長石とカリ長石は
頻度の分布幅が広く、異方的な性質が残ることを表している。
このことは、カリ長石やナトリウム長石の粒内や粒界で強い散乱が起きる
ことを意味し、長距離を伝播した弾性波の散乱減衰のメカニズムを理解する上で重要な情報になる。
%
%
%
ここで,実験で得られた位相速度$c_p^{(\alpha)}$と単結晶の位相速度$v$の比較を行う.
単結晶の位相速度の計算には,文献に与えられた弾性係数$C_{ijkl}$と密度$\rho$
を用い,Christoffelの方程式を解くことによって求める.
Christoffelの方程式は,平面波の振動および進行方向を表す単位ベクトルを
それぞれ
\begin{equation}
	\fat{d}=(d_1,\, d_2,\, d_3), \ \ 
	\fat{p}=(p_1,\, p_2,\, p_3)
	\label{eqn:unit_vecs}
\end{equation}
とすれば、次のように表すことができる.
\begin{equation}
	\left( C_{ijkl}p_j p_k - \rho v^2 \right)d_k=0
	\label{eqn:Christoffel}
\end{equation}
式(\ref{eqn:Christoffel})を与えられた進行方向ベクトル$\fat{p}$
に対して解くことで,$\fat{p}$方向への疑似縦波と疑似横波の
位相速度$v$を求めることができる.ここでは,$\fat{p}$を
\begin{equation}
	\fat{p}=\left(
		\sin\theta \cos\phi, \,
		\sin\theta \sin\phi, \,
		\cos\theta
	\right)
	\label{eqn:p_polar}
\end{equation}
球座標表示し,
\begin{equation}
	0 \leq \theta \leq \pi, \ \ 
	0 \leq \phi \leq 2\pi
	\label{eqn:}
\end{equation}
の範囲において,$v(\theta, \phi)$を1度刻みで求めた.
図\ref{fig:***}は,その結果を立体角$\Delta \Omega = \sin\theta \Delta \theta \Delta phi$
で重み付けした頻度分布として示したものである.これらは,多数の単結晶が一様な方位分布をもつと
仮定したときの,縦波位相速度の分布を表す.
実験で求めた鉱物粒の位相速度$c^{(\alpha)}_p$と単結晶の位相速度分布を比較すると,
単結晶の位相速度は花崗岩の鉱物粒に対する結果よりも平均的に大きな値を取ることが分かる。
これは,花崗岩では鉱物粒に含まれるマイクロクラックによって軟化し,理想的な単結晶よりも
位相速度が低下するためと解釈することができる.
位相速度の分布幅についてみれば,鉱物種間での大小関係は単結晶と花崗岩とも
カリ長石>ナトリウム長石>石英の順で変わりはない.
ただし,分布形状は単結晶と花崗岩の場合で大きく異なっている.
例えば,石英の場合,花崗岩に対する結果では分布幅は狭く,単結晶に比べて異方性が弱められている.
このことは,石英中のマイクロクラックには方向性が弱く,その結果単結晶の場合よりも等方的に振る舞う
ことを示唆している。一方、カリ長石とナトリウム長石は単結晶と比べて位相速度が低下する点は
石英と共通するが,分布幅は石英に比べて大きくクラックの存在下でも異方性を示す。
特にカリ長石の位相速度は分布幅が広く,これは、へき開の存在により著しい軟化が起きるものの,
へき開の方向性のために強い異方性が残ると考えれば解釈がつく.
\begin{figure}
\begin{center}
	\includegraphics[clip,scale=0.5]{Figs/hist_Q.eps}
	\caption{位相速度の分布(石英).}
	\label{fig:fig10}
\end{center}
	\vspace{-10mm}
\end{figure}
\begin{figure}
\begin{center}
	\includegraphics[clip,scale=0.5]{Figs/hist_N.eps}
	\caption{位相速度の分布(ナトリウム長石).}
	\label{fig:fig11}
\end{center}
	\vspace{-10mm}
\end{figure}
\begin{figure}
\begin{center}
	\includegraphics[clip,scale=0.5]{Figs/hist_K.eps}
	\caption{位相速度の分布(カリ長石).}
	\label{fig:fig12}
\end{center}
	\vspace{-10mm}
\end{figure}
\begin{figure}
\begin{center}
	\includegraphics[clip,scale=0.5]{Figs/hist_all.eps}
	\caption{位相速度の頻度分布(全データを使用).}
	\label{fig:fig13}
\end{center}
	\vspace{-10mm}
\end{figure}
\begin{figure}
\begin{center}
	\includegraphics[clip,scale=0.5]{Figs/monocrystal.eps}
	\caption{位相速度の頻度分布(単結晶の場合).}
	\label{fig:fig14}
\end{center}
	\vspace{-10mm}
\end{figure}
%%%%%%%%%%%%%%%%%%%%%%%%%%
