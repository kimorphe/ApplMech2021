本節では,実験で得られた鉱物種毎の位相速度の頻度分布を考慮して行った波動伝播解析の結果を示す.
ここでは簡単のため,位相速度のコントラストによって鉱物粒を表現することで非均質等方弾性体
として花崗岩を模擬し,2次元スカラー波の伝播散乱解析を行う.その際,位相速度の空間分布を
鉱物種をごとに異なる頻度分布に基づいて設定した場合と,鉱物種による区別を行うことなく単一
のガウス分布に従うものとして与えた2種類の場合について行う.両者の結果を比較することにより,
鉱物種毎に位相速度を設定することには意義があることを例示する.
\subsection{問題設定}
図-\ref{fig:fig15}(a)のような幅$W$高さ$H$の矩形領域$D$における波動伝播問題を考える.
領域$D$内の位相速度$c_p(\fat{x})$と密度$\rho$は,予め与えられているものとする.
ここで,密度$\rho$は鉱物種による差が小さいことから,以下では全ての鉱物種に対して
一定値で近似する.いま,領域$D$の左側の境界($x=0$)に一様な応力$\sigma_{in}(t)$を加え,
$x>0$方向に伝播する平面波を励起する.境界に加える応力の時間変化は
図-\ref{fig:fig15}(b)のものを用いる.これは,周波数1[MHz]のコサイン波をガウス関数で
振幅変調した波形である.物体が占める領域は$y$方向には十分長い場合を想定し,
$D$上下の境界$y=0,H$には周期境界条件を課す.一方,右側の境界$x=W$では応力はゼロであるとして,
このとき$D$内各所で散乱を起こしながら伝播する平面波の挙動を調べる.
媒体は非均質であるため,厳密な意味での平面波は存在し得ない.
しかしながら,以下で見るように,位相の揃った初動部分は概ね$x$方向への1次元的な
進行波と見ることができるため,ここでは便宜的に"平面波"という呼び方をする.
\begin{figure}
\begin{center}
	\includegraphics[clip,scale=0.5]{Figs/model.eps}
	\caption{(a)計算領域$D$,(b)加振波形$\sigma_{in}(t)$,および(c)差分格子の配置.}
	\label{fig:fig15}
\end{center}
\end{figure}
\subsection{波動伝播問題の定式化および数値解析法}
スカラー波動場の支配方程式は,等方線形弾性体の運動方程式において
せん断剛性$\mu$をゼロとすることで得られ,次のように表すことができる.
\begin{equation}
	\rho \dot{\fat{v}}=\nabla \sigma
	\label{eqn:eq_mot}
\end{equation}
\begin{equation}
	\dot{\sigma}=\lambda \nabla \cdot \fat{v}
	\label{eqn:Hooke}
\end{equation}
ここに,$\fat{v}=\fat{v}(x,y,t)$は速度ベクトルを,$\sigma=\sigma(x,y,t)$は
応力テンソルの球成分を,$\lambda=\lambda(x,y)$はラメ定数を表す.
位相速度とラメ定数の関係は
$\lambda =\rho c_p^2$であり,位相速度と密度が指定されればラメ定数を与えることができる.
媒体は時刻$t=0$において静止状態にあるものとし,初期条件を
\begin{equation}
	\sigma(x,y,0)=0, \ \ \fat{v}(x,y,0)=\fat{0}, \ \ (x,y) \in D
	\label{eqn:IC}
\end{equation}
とする.境界条件は上に述べた問題設定より,次のように書くことができる.
\begin{equation}
	\sigma(0,y,t)=\sigma_{in}(t), \ \ t>0
	\label{eqn:}
\end{equation}
\begin{equation}
	\sigma(W,y,t)=0, \ \ t>0
	\label{eqn:}
\end{equation}
\begin{equation}
	\sigma(x,0,t)=\sigma(x,H,t),  \ \ t>0
	\label{eqn:}
\end{equation}
以上で表される初期値-境界値問題を解くことで,領域$D$における平面波の伝播,散乱挙動を
調べることができる.そのための数値解析法には有限差分法を用いる.
図-\ref{fig:fig15}(c)に,差分法解析のための空間格子配置を示す.
この図は差分格子の単位セルを示したもので,丸が速度ベクトル$\fat{v}$,
ひし形が応力$\sigma$の計算格子を表す.非均質媒体を扱うためにはこれらの格子点に加え,
密度$\rho$と位相速度$c_p$を与える格子を決めておく必要がある.ここでは,位相速度は
応力の計算格子において,密度は速度の計算格子において与える.
ただし,密度に関しては一定値で近似しているため,どの格子点に密度評価点を
対応付けても違いはない.このようにすることで,位相速度の不連続部を数値解析上平均化する
こと無く取り扱うことができる.式(\ref{eqn:eq_mot})と(\ref{eqn:Hooke})の空間微分は,図-\ref{fig:fig15}(c)
の格子上で中央差分によって近似する.時間微分は,リープフロッグ法に基づいて離散化し,
速度$\fat{v}$と応力$\sigma$を交互に陽解法で求める.
\subsection{解析条件}
領域$D$のサイズを
\begin{equation}
	W=50[{\rm mm}], \ \  H=30[{\rm mm}]
	\label{eqn:dom_size}
\end{equation}
とし,時刻$ 0 <t\leq 20[\mu$sec]における解$\fat{v}$と$\sigma$を差分法によって求める.
位相速度の空間分布は,実験で得た頻度分布(図\ref{fig:fig13})から鉱物種毎にサンプリングし,
鉱物粒内では一定値として与えた.このようにして設定した位相速度の空間分布の一例を
図-\ref{fig:fig16}(a)に示す.
この図は領域$D$における位相速度$c_p(\fat{x})$の値をカラーマップとして示したものである.
なお,鉱物粒分布は以下のようにして与えた.
はじめに,花崗岩供試体のスキャナ画像から解析領域$D$に相当する範囲を切り出し,
画像ピクセル毎に鉱物種を特定する.次に,ピクセル間のユークリッド距離に基づいて
鉱物種ごとにクラスタリングを行う.その結果,同一クラスターに属すると判定された
ピクセルを一つの鉱物粒として扱う.クラスタリングを鉱物種毎に行うことで,
石英,ナトリウム長石,カリ長石の結晶粒をそれぞ生成することができる.
なお,黒雲母の位相速度データは得られていないため,黒雲母と判定させれたピクセルは
最近傍のクラスターに振り分け,そのクラスターの鉱物種で代替する.
この方法ではクラスター数を調整することで,結晶粒の数を任意に調整できる.
またその結果として,平均的な結晶粒サイズもコントロールできる.
さらに,クラスターの初期位置を乱数を使って設定することで,多数のランダム媒体を
簡単に生成できる.ここでは,クラスター数は,それぞれの鉱物種について200個,
合計600の鉱物粒を生成した.そのため,結晶粒の粒径はおよそ1.6mm程度となっている.
なお,スキャナの画像上で特定した鉱物種は,黒雲母を除き事後的に変更されることはない,
クラスタリングによって鉱物分布自体が変わることはない.鉱物粒と位相速度マップの作成に用いた
画像は,そのまま差分法解析の格子としても用いる.
そのため,差分格子の間隔$\Delta x$と$\Delta y$は,1画素に相当するサイズとした.
一方,時間ステップ$\Delta t$は,位相速度の最大値$c_{max}$に対して,安定条件を
満足するように決定した.その結果,差分法解析におけるステップ長は,
以下のように与えることとした.
\begin{equation}
	\Delta x = \Delta y= 0.055[{\rm mm}], \ \ \Delta t=0.04[\mu{\rm sec}]
	\label{eqn:}
\end{equation}
\subsection{解析結果}
以上の方法で行った差分法解析の結果を図-\ref{fig:fig16}の(b)と(c)に示す.
これらの図は計算領域$D$における速度場の絶対値$\left| \fat{v}\right|$
をカラーマップとして表示したもので,(b)は時刻$t=4.0[\mu$sec],(c)は
$t=8.0[\mu$sec]におけるスナップショットである.
ここには領域左端で励起された波が,散乱波を励起しながら$x>0$の方向に進行する
様子が示されており,赤の矢印で示した箇所には大きな振幅を持つはっきりした
波面が現れていることが分かる.この波面は,入力に用いた応力波形$\sigma_{in}(t)$の
時刻$t=1.5[\mu$s]におけるピークに対応し,位相速度の平均値5.4[km/s]
程度のスピードで進行していることが確かめられる.これらの波面の背後(x<0の側)には
複雑な多重散乱場が形成され,初動部分が通過した後も振動を続けることが分かる.
また,矢印で示した波面は伝播につれ,屈曲の程度はあまり変化しないが,
波面上での振幅変動が大きくなっている.
\subsection{解析結果}
ここでは応力とひずみの関係にはフックの法則が用いられているため,
材料減衰(散逸)の効果は含まれない.また,$y$軸方向には周期境界条件を課し,
波動場を領域左端を一様に加振して励起しているため幾何減衰も生じない.
従って,振幅の低下は散乱減衰のみによって生じているため,初動部分の振幅変化を
見ることで散乱減衰挙動を調べることができる.そこで以下では,伝播距離に応じた速度振幅
の変化を見ることで,花崗岩中の弾性波がどのような散乱減衰則に従うかを検討する.
花崗岩は本質的にランダムな不均質媒体である.そのため,特定の点で得られた波形に
意味を見出すことは難しく,結果の解釈は何らかの平均や偏差といった統計的な見方に基づいて
行う必要がある.そこで,速度ベクトル$\fat{v}=(v_1,v_2)$の
$x$成分$v_1(x,y,t)$を$y$座標に関して積分して得られる平均波形:
\begin{equation}
	\bar v_1(x,t):=\frac{1}{H}\int_0^Hv_1(x,y,t)dy
	\label{eqn:v1_ave}
\end{equation}
を用いて平均的な距離減衰挙動を調べる.なお,減衰は一般に周波数にも依存するため,
式(\ref{eqn:v1_ave})のフーリエ変換をとり,周波数$f=1$[MHz]における
フーリエ振幅:
\begin{equation}
	\bar V_1(x):={\cal F} \left. \left\{ v_1(x,t)\right\} \right|_{f=1{\rm MHz}} 
%	:=\int \bar v_1(x,t)e^{2\pi f t}dt
	\label{eqn:}
\end{equation}
の位置$x$による変化を見ることにす.更にこの作業を10個の異なるモデルを用いて行い,
その平均$\left< \bar V_1 \right>(x)$をとる.
これを$x=0$における値で規格化し,速度振幅比を
\begin{equation}
	\alpha(x) := 
	\left|
	\frac
	{ \left< \bar V_1(x) \right>}
	{ \left< \bar V_1(0) \right>}
	\right|
	\label{eqn:decay_x}
\end{equation}
と定義し,位置$x$との関係から減衰挙動を調べる.なお,
$\left<\bar V_1\right>(x)$の評価に用いる10個の波動解析モデルは,
結晶粒を設定する際のクラスタリングにおいて,クラスターの初期位置を
ランダムに変化させて生成した.

図-\ref{fig:fig16}に,青の実線で振幅比$\alpha(x)$の計算結果を示す.
$\alpha(x)$は$x$の増加につれ,細かく振動しながら減少している.
この振動成分は,領域内各所で発生する後方散乱波と$x$の正方向に伝播する
波動成分が干渉することによって生じたものと考えられる.
一方,大域的な$\alpha$の減少は,当初位相の揃っていた初動部分のもつエネルギーが,
散乱によって不規則に分散されることによって生じている.
$\alpha$の減少は概ね直線的だが,興味深いことに$x=20$[mm]付近で折れ曲がり,
$x>20$[mm]以遠では減少がやや緩慢になっている.
この正確なメカニズムを解明することは今後の課題だが,
減衰が$y$方向の波数$k_y$に依存することに起因したものと
予想される.完全な平面波では伝播直角方向($y$方向)の波数は零だが,
不均質媒体では波面が屈曲するために波数$k_y$は複数の波数成分をもつことになる.
これらの中に減衰のオーダが異なるものが混在と考えれば,図-\ref{fig:fig17}
のように減衰挙動が切り替わるような振る舞いが見られることにも不思議では無い.
このような挙動はより一般のランダム不均質媒体でもみられるか否かは興味ある問題である.
そこで,鉱物種を区別することなく位相速度を割り当てたモデルで波動伝播解析を
行い,その結果を用いて$\alpha$の評価を行ってみる.
ただし,結晶粒分布のモデルはこれまでと同じものを用い,
位相速度の割当だけを鉱物種を区別せず単一の正規分布に従うとして行う.
その際,正規分布の平均と標準偏差には,表\ref{tbl:tbl1}に示した
位相速度の全データに対する平均と標準偏差を用いた.
図-\ref{fig:fig17}の赤の実線は,このようにして作成したモデルから評価した
振幅比$\alpha$を示すものである.この場合,細かな振動があることは鉱物種を区別した場合と同様だが,$\alpha$は
ほとんど一定の傾きをもって推移しており,減少率が切り替わるような挙動は現れていない.
従って,$x=20$[mm]付近に見られた減少率の変化は,鉱物種を区別してモデルを作成
したことに起因した,何らかの規則性を反映したものと考えることができる.
またこのような差が生じることは,花崗岩のような多結晶質体中の波動伝播メカニズムを理解するためには,
鉱物種の特徴を反映したモデルを用いる必要があることを意味している.
%本研究で示した位相速度の頻度分布はそのような目的に資するはじめての統計データであると言える.
\begin{figure}
\begin{center}
	\includegraphics[clip,scale=0.5]{Figs/field.eps}
	\caption{(a)数値解析モデルに設定した位相速度.(b),(c)速度場のスナップショット.}
	\label{fig:fig16}
\end{center}
\end{figure}
\begin{figure}
\begin{center}
	\includegraphics[clip,scale=0.5]{Figs/decay.eps}
	\caption{振幅比$\alpha$の距離$x$に対する変化.}
	\label{fig:fig17}
\end{center}
\end{figure}
