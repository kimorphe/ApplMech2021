%\section{超音波計測実験}
本節では,花崗岩の局所的な弾性波速度を鉱物種毎に調べることを目的として行った,
超音波計測について述べる.実験には,薄板状の花崗岩供試体を用い,
板厚方向に透過する縦波超音波を計測した.以下,実験手法の詳細を示す.
\subsection{実験供試体}
超音波計測に用いた花崗岩供試体を図\ref{fig:fig1}に示す.
この供試体は岡山県万成地域の採石場で採取した万成花崗岩を,
厚さ3.42㎜の平板に切断加工したものである.供試体表面は研磨等による
仕上げは行っていないが,表面に凹凸や目視で認められるような欠けや割れ,
明らかな風化は認められないことを確認している.
図\ref{fig:fig1}-(a)は超音波の送受信を行う供試体の表面の全体を,
(b)はその一部を拡大して鉱物粒の状況を示したものである.
図\ref{fug:fig1}-(b)に記入したように,供試体を構成する主要鉱物は石英(Qt),
カリ長石(K),ナトリウム長石(Na)と黒雲母(Bt)の4種類で,各々の含有割合は
カリ長石34\%,ナトリウム長石17\%,石英44\%,雲母5\%である.
また,これら鉱物種の平均粒径は順に約1.6,0.9,1.1および0.5mmである.
ここでの目的は,板厚方向に透過する縦波超音波を計測し,
鉱物種毎の位相速度を求めることにある。
そのためには厚み方向に鉱物種が変化しないよう,供試体の厚みを鉱物粒径より小さく
することが望ましい.一方,板厚が小さい場合,直接透過波と多重反射波成分が時間軸上
で近接して観測されて両者の分離が困難になる.従って,超音波計測の点では計測に
用いる入射波波長が利用可能な試験片厚さの限界を定める.以下に述べるように,
本研究では1MHzの超音波センサーを用いる.花崗岩の縦波速度はおよそ5$\sim$6km/s程度で,
1MHzでの波長は5$\sim$6mm程度である.従って,6mmの波長を持つ透過波1波を時間軸上で反射波
から分離して観測するためには3mm以上の板厚が必要になる.このことに加え,
供試体を切断加工して厚みを一定にできること,供試体が実験中に破損することのない
程度の強度を持つことを条件として考慮した結果,ここでは3.5mm程度の厚さを
もつ供試体を作成することとした.なお,実際に切断加工を行って作成した
結果,供試体の出来上がり寸法での厚さは3.42mmであった.
\subsection{超音波計測装置の構成}
実験に用いた超音波計測装置の構成を図-\ref{fig:fig2}に示す.
計測系は超音波を送信するための圧電探触子と花崗岩供試体に加え,
3軸ステージ,レーザードップラー振動計(LDV),オシロスコープ,
および高周波スクウェア−ウェーブパルサーで構成されている.
花崗岩供試体は水平2軸,回転1軸の3軸ステージ上に固定し,LDVによるレーザー照射位置を精確に調整する.
その際,送信探触子は供試体の下面に接触させて固定し供試体とともに移動させる.
探触子の駆動はスクウェア−ウェーブパルサーを用いて行い,100Vの矩形パルス電圧を印加する.
受信にはLDVを用い,供試体上面の鉛直動を計測する.
LDVで受信した信号はオシロスコープに転送し,4,096回の平均化を行った後,
デジタル波形としてPCで収録する.このとき,サンプリング周波数は15MHz,
計測時間範囲は100$\mu$秒とし,全ての位置での計測を同じ条件で行った.
超音波の送信には線集束型の圧電探触子を用い,供試体内に円筒波を励起した,
個々で用いた探触子は曲率半径26.1mm,投影面積25mm×40mmの圧電素子をくさび状の
ポリスチレンシューに取り付けたもので,シュー先端部の幅と長さはそれぞれ2mmと50mm,
共振周波数は1MHzである.
\subsection{送受信位置}
図\ref{fig:fig3}に送信および受信点の配置を示す.
図中の${\cal R}(X)$で示した線は送信センサー直上にとった長さ40mmの観測線を表し,
$X$はその位置($x$座標)を意味する.超音波波形の計測は,${\cal R}(X)$上をピッチは0.25mmで
スキャンして行い,1観測線あたり161点で透過波波形を取得した.
このような透過波計測を$x$方向に1mm間隔で送信センサーを移動させ,
$10\leq X \leq 59$[mm]の50の観測線について行うことで合計80,500の波形を取得した.
各観測点における鉱物種は供試体表面をスキャナで撮影したデジタル画像上で特定し,
鉱物種毎の統計を取ることができるようにする.
図\ref{fig:fig4}は鉱物種の特定を行った結果を示したもので,
(a)はスキャナ画像を,(b)は各画像ピクセルで特定した鉱物種を塗り分けて表示したものである。
鉱物種の特定は画像編集ソフト上で色彩に基づき手作業で行った.
このような作業は,供試体の上面(観測側)と下面(送信側)の両方で行い,
表裏面で異なる鉱物種となっている観測点での波形データは除外し,
送信と観測側で同一の鉱物種と判定された一で得られた波形だけを鉱物種毎の
位相速度を算出するために用いることとした.その結果,黒雲母は6点,石英590点,
カリ長石825点,ナトリウム長石549点でのデータが最終的に得られた.
黒雲母の含有量は他の造岩鉱物に比べて小さくデータ数が少ないことから,
以下では黒雲母を除く3つの鉱物の位相速度について検討を行う.
%%%%%%%%%%%%%%%%%%%%%%%%%%%%%%%%%%%%%%%%%%%%%%%%%%%%%%%%%%%%%%
\begin{figure}
\begin{center}
\includegraphics[clip,scale=0.35]{Figs/samples.eps}
\caption{
	超音波計測に用いた花崗岩供試体.
	K,Na,Qt,Btは,それぞれカリ長石,ナトリウム長石,石英および黒雲母を意味する.
}
\label{fig:fig1}
\end{center}
	\vspace{-5mm}
\end{figure}
\begin{figure}[t]
\begin{center}
\includegraphics[clip,scale=0.45]{Figs/ut_setup.eps}
\caption{ 超音波計測装置の構成. }
\label{fig:fig2}
\end{center}
%	\vspace{-15mm}
\end{figure}
\begin{figure}[t]
\begin{center}
\includegraphics[clip,scale=0.45]{Figs/cod.eps}
\caption{
	超音波の送信位置と観測線$\cal R$の配置.
}
\label{fig:fig3}
\end{center}
%	\vspace{-15mm}
\end{figure}
\begin{figure}[t]
\begin{center}
\includegraphics[clip,scale=0.5]{Figs/map.eps}
\caption{
	画像ピクセル毎に特定した鉱物種の分布
	(緑:石英,赤:カリ長石,白:ナトリウム長石,黒:黒雲母).
}
\label{fig:fig4}
\end{center}
%	\vspace{-15mm}
\end{figure}
%%%%%%%%%%%%%%%%%%%%%%%%%%%%%%%%%%%%%%%%%%%%%%%%%%%%%%%%%%%%%%

