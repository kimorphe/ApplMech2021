花崗岩はき裂性の結晶質岩で,新鮮な状態でも造岩鉱物粒程度のスケールでは多数のマイクロクラックを有することが知られている.
マイクロクラックは花崗岩の透水性や物質輸送特性,剛性,強度,物理・化学的な風化の進行に影響を与える.
そのため,マイクロクラックの大きさや量を把握することは,高レベル放射性廃棄物の地層処分のように,
長期に渡って安全性を担保する必要のある処分場を花崗岩体中に建設する際に重要となる.
岩石中のき裂の大きさや量,分布状況は偏光顕微鏡で岩石薄片を観察することで調べられる.
しかしながら,岩石薄片の作成には手間がかかる上,岩石内部の状態やき裂の経時的な変化を観察することはできない.
一方,岩石の弾性波試験は結果の定量的な解釈が難しいことが多いものの,試料内部の状態を簡単な検査で調べることができ,
経時的なモニタリングも容易にできるという面で利点がある.
さらに,弾性波の伝播はき裂の影響を強く受けるため,弾性波計測結果からき裂の量や成長を調べる試みが行われている[1].
ただし,花崗岩中の弾性波伝播は鉱物粒やき裂による多重散乱のために非常に複雑で,
き裂の定量的な評価や,組成の異なる花崗岩間での比較行うには至っていない. 
 花崗岩中の弾性波伝播特性を理解するためには,き裂を含む造岩鉱物粒による弾性波の多重散乱解析を行うことが有効と考えられる.
多重散乱解析の実施には各造岩鉱物の密度と弾性係数値が必要とされる.
このうち密度は物質の結晶構造だけで決まるため鉱物種ごとに定めることができ,
また鉱物種間での差も小さい.
一方,弾性係数は結晶軸方向やき裂の有無による不確実性があるため,
対象とする岩石の状態を反映して統計的に与える必要がある.各種造岩鉱物の弾性係数はよく知られているが,
それらは無欠陥の単結晶に対するもので,文献値だけから造岩鉱物の弾性係数を定めることはできない.
そこで本研究では,超音波計測によって局所的な弾性波速度を直接求める.
これを多数の鉱物粒について行うことで弾性波速度の統計データを得る.
以下では,この目的のために行った超音波計測方法について述べた後,縦波の位相速度についてについて得られた結果を報告する.
具体的には,0.5MHzから1.5MHz程度の周波数帯域での位相速度の確立分布を示す.
その結果,結晶粒のスケールでは鉱物種に依らず速度分散はほとどんど無いが,
長石と石英では弾性波速度の分布が有意に異なることを示す.また,これら実験で取得した弾性波速度を
用た数値波動解析により,縦波平面波の散乱減衰挙動が,位相速度の分布を正規分布で与えた場合と
,計測から得たものを用いた場合で異なることを示す.これらの結果から,弾性波媒体として花崗岩をモデル化する上
で,実際に計測でえた速度分布モデルを用いることが必要であることを述べる.

