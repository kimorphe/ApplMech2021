%%#!platex
%
% Example of Japanese Paper of JSCE
% for LaTeX2e users
%
% revised on 4/25/2014
%
%%%%%%%%%%%%%%%%%%%%%%%%%%%%%%%%%%%%%%%%%%%
%
% もし jis フォントメトリックを使う場合は,以下をアンコメントしてください.
% \DeclareFontShape{JY1}{mc}{m}{n}{<-> s * jis}{}
% \DeclareFontShape{JY1}{gt}{m}{n}{<-> s * jisg}{}
%
\documentclass{jsce}
%
\usepackage{epic,eepic,eepicsup}
%\usepackage{graphicx,multicol}
\usepackage{graphicx}
\usepackage{multicol}
\usepackage{amsmath}
%\usepackage{showkeys}
\usepackage{setspace}
%  amsを使う方は以下をアンコメントしてください.
%\usepackage{amssymb,amsmath}
% 英語はサポートしているかどうか不明
% \inenglish
% 学会サンプルに times とあるので指定しておきます
\usepackage{times}
%
\finalversion
\pagestyle{empty}
%
\title{
	花崗岩の造岩鉱物粒スケールでみた\\
	弾性波伝播特性
}%
\endtitle{
ELASTIC WAVE PROPAGATION IN A GRANITE AT A SCALE OF ROCK FORMING MINERAL GRAINS
}
%
% emailアドレスのフォントをタイプライター体にしたい方は次行をアンコメント
% \emailstyle{\ttfamily}
% emailアドレスを公開される方は,
%% \thanks{○○○○○○\email{your_name@foo.ac.jp}}のようにしてください.
%
\author{木本 和志\thanks{正会員 博士(工学) 
岡山大学 学術研究院環境生命科学学域
(〒700-8530 岡山県岡山市北区3丁目1番地1号)\email{kimoto@okayama-u.ac.jp} (Corresponding Author)}・
岡野 蒼\thanks{学生会員 岡山大学環境生命科学研究科 (〒700-8530 岡山県岡山市北区3丁目1番地1号)}・
斎藤 隆泰\thanks{正会員 博士(工学)群馬大学大学院理工学府 環境創生部門(〒376-8515 群馬県桐生市天神町 1-5-1)}
%佐藤 忠信\thanks{正会員 博士(工学)神戸学院大学・現代社会学部・防災社会学科(〒650-8586神戸市中央区港島1-1-3)}・
%松井 裕哉\thanks{正会員 修士(工学)日本原子力研究開発機構・幌延深地層研究センター・堆積岩処分技術開発Gr
%(〒098-3224 北海道天塩郡幌延町北進432番地2)}
}
\endauthor{Kazushi KIMOTO, Aoi OKANO, Takahiro SAITOH
%\\ Tadanobu SATO and Hiroya MATSUI 
}
%
\abstract{
\small
本研究は,花崗岩における縦波の局所的な伝播挙動を調べたものである.具体的には,造岩鉱物粒スケールでの位相速度を推定することを目的に,花崗岩供試体の板厚方向に透過する縦波をレーザードップラー振動計で計測を行った.位相速度の評価は別途取得した参照波形と透過波形の位相差に基づいて行い,得られた位相速度を造岩鉱物種毎に統計的に処理して頻度分布を求めた.その結果,カリ長石の位相速度は非対称かつ幅の広い分布に従う一方,石英の位相速度は対称かつより狭い分布に従うことが分かった.さらに,以上で得た頻度分布を用いて花崗岩の2次元数値解析モデルを作成して平面波伝播解析を行ったところ,ガウス分布によって摂動された位相速度分布を与えた解析モデルとは異なる,特異な散乱減衰の挙動が現れることが示された.
}
%
\keywords{granite, rock-forming mineral, ultrasonic wave, phase velocity, probability density}
%
\endabstract{% Yes blank line
\normalsize
This study investigates the propagation characteristics of P-wave in granite. Specifically, the local phase velocity is evaluated experimentally using a granite plate. In the experiment, ultrasonic P-waves transmitted in the thickness direction are measured by scanning the plate surface with a laser Doppler vibrometer. The phase velocity is evaluated based on the phase delay relative to an independently measured reference signal. The phase velocity data obtained thus are processed statistically to produce probability density functions for each rock-forming mineral species. As a result, a highly asymmetric and broad probability density function (PDF) is obtained for the P-wave velocity of potassium feldspar, while the PDF for quartz is found to be symmetric and much narrower. Finally, the PDFs are used to generate 2-dimensional numerical models of granite on which plane wave propagation analyses are performed. The numerical results showed that the phase velocity field perturbed according to the experimentally obtained PDFs gives a peculiar decay profile compared to the Gaussian perturbed models.
}
%
% \titlepagecontrol{1}
%
%\receivedate{2019.7.19}
% \receivedate{January 15, 1991}
%
% \def\theenumi{\alph{enumi}}  % もし enumerate 最初の箇条を (a) と
% \def\labelenumi{(\theenumi)} % したい場合・・・
%
\begin{document}
\maketitle
%%%%%%%%%%%%%%%%%%%%%%%%%%%%%%%%%%%%%%%%%%%%%%%%%%%%%%%%%%%%%%%%%%%%%%%%%%%%%
\section{はじめに}
	花崗岩はき裂性の結晶質岩で,新鮮な状態でも造岩鉱物粒程度のスケールでは多数のマイクロクラックを有することが知られている\cite{Kudo1-Takagi}.
マイクロクラックは花崗岩の透水性や物質輸送特性,剛性,強度,物理・化学的な風化の進行に影響を与える.
そのため,マイクロクラックの大きさや量を把握することは,高レベル放射性廃棄物の地層処分のように,
長期に渡って安全性を担保する必要のある処分場を花崗岩体中に建設する際に重要となる.
岩石中のき裂の大きさや量,分布状況は偏光顕微鏡で岩石薄片を観察することで調べられる.
しかしながら,岩石薄片の作成には手間がかかる上,岩石内部の状態やき裂の経時的な変化を観察することはできない.
一方,岩石の弾性波試験は結果の定量的な解釈が難しいことが多いものの,試料内部の状態を簡単な検査で調べることができ,
経時的なモニタリングも容易にできるという面で利点がある.
さらに,弾性波の伝播はき裂の影響を強く受けるため,弾性波計測結果からき裂の量や成長を調べる試みが行われている\cite{Griffiths}.
ただし,花崗岩中の弾性波伝播は鉱物粒やき裂による多重散乱のために非常に複雑で,
き裂の定量的な評価や,組成の異なる花崗岩間での比較行うには至っていない. 
 花崗岩中の弾性波伝播特性を理解するためには,き裂を含む造岩鉱物粒による弾性波の多重散乱解析を行うことが有効と考えられる.
多重散乱解析の実施には各造岩鉱物の密度と弾性係数値が必要とされる.
このうち密度は物質の結晶構造だけで決まるため鉱物種ごとに定めることができ,
また鉱物種間での差も小さい.
一方,弾性係数は結晶軸方向やき裂の有無による不確実性があるため,
対象とする岩石の状態を反映して統計的に与える必要がある.各種造岩鉱物の弾性係数はよく知られているが,
それらは無欠陥の単結晶に対するもので,文献値だけから造岩鉱物の弾性係数を定めることはできない.
そこで本研究では,超音波計測によって局所的な弾性波速度を直接求める.
これを多数の鉱物粒について行うことで弾性波速度の統計データを得る.
以下では,この目的のために行った超音波計測方法について述べた後,縦波の位相速度についてについて得られた結果を報告する.
具体的には,0.5MHzから1.5MHz程度の周波数帯域での位相速度の確立分布を示す.
その結果,結晶粒のスケールでは鉱物種に依らず速度分散はほとどんど無いが,
長石と石英では弾性波速度の分布が有意に異なることを示す.また,これら実験で取得した弾性波速度を
用た数値波動解析により,縦波平面波の散乱減衰挙動が,位相速度の分布を正規分布で与えた場合と
,計測から得たものを用いた場合で異なることを示す.これらの結果から,弾性波媒体として花崗岩をモデル化する上
で,実際に計測でえた速度分布モデルを用いることが必要であることを述べる.


\section{超音波計測実験}
	%\section{超音波計測実験}
Fig.1に実験に用いた花崗岩試験片を示す.この試験片は岡山県万成地域の採石場で採取した万成花崗岩を厚さ3.42㎜の平板状に切断加工したものである.この試験片を用い,板厚方向へ透過する縦波を計測して位相速度と減衰定数を評価した.超音波の送信は供試体表面に接触させた線集束型の圧電超音波探触で,受信はレーザードップラー振動計(LDV)を用いて行った.送信探触子は曲率半径26.1mm,投影面積25mm×40mmの圧電素子をくさび状のシュー材に取り付けたもので,シュー先端部の幅と長さはそれぞれ2mmと50mm,共振周波数は1MHzである.試験片は水平2軸,回転1軸の3軸ステージ上に固定して位置を精確に調整し,探触子は100Vの矩形電圧パルスを印加して駆動した.LDVでの受信波形はオシロスコープ上で4,096回平均化した後,デジタル波形として収録し,その際,サンプリング周波数は15MHz,計測時間範囲は100μ秒とした.Fig.2に送信および受信点の配置を示す.図中のRで示した点線は送信センサー直上にとった長さ40mmの測線を表している.R上でのスキャンピッチは0.25mmでR上161の観測点で波形を取得した.このような透過波計測をy方向に1mm間隔で送信センサーを移動させて計40測線について行うことで合計6,460の波形を取得した.なお,透過波が通過した鉱物種はデジタル画像上で特定し,鉱物種毎の統計を取ることができるようにする.ただし,試験片の表裏面で異なる鉱物種となっている観測点での波形データは除外し,表裏面で同一の鉱物種となっている点で得られた波形だけを使用することとした.その結果,黒雲母では6点,石英590点,カリ長石825点,ナトリウム長石549点でのデータが最終的に得られた.黒雲母の含有量は他の主要造岩鉱物に比べて小さくく,波形数も非常に少ないことから,本研究では黒雲母を除く3つの鉱物について検討を行うこととした.
\subsection{実験供試体}
実験に用いた花崗岩供試体を図-\ref{fig:fig1}に示す.
この供試体は,岡山県万成地域の採石場で採取した万成花崗岩を岩石カッターでブロック状に
切断加工したものである.研磨等による切断面の仕上げは行っていないが,供試体表面に凹凸や目視で認められる欠けや割れ,明らかな風化はない.
万成花崗岩の主要造岩鉱物は,カリ長石,ナトリウム長石,石英および雲母の四種類で,
特徴的な桃色の色合いをした箇所がカリ長石である.
これら主要鉱物の割合は,カリ長石34\%,ナトリウム長石17\%,石英44\%,雲母5\%で,平均粒径は順に,
約1.6,0.9,1.1および0.5mmである.いずれの鉱物も音響異方性を示し,結晶軸からの方向に
応じて弾性波速度が変化する.ただし,供試体のほぼ95\%を占める長石と石英の弾性波速度は,
縦波が5.59$\sim$6.06km/s,横波が3.06$\sim$4.11km/sとのデータが知られている\cite{RockPhys}.
試験片のサイズは長さ$L=178$mm, 幅$W=56$mm,厚さ$H=30$mmで,
計測位置は図-\ref{fig:fig1}のような$xyz$座標系で表す.
超音波の送信と受信は,試験片の上面$(z=0)$mmにおいて行い,$x$軸の正方向へ伝播する表面波を計測する.
また,均質材における波動伝播挙動との比較を行うため,同様な計測を,アルミニウムブロック供試体でも行う.
アルミニウム供試体のサイズは,長さ200mm,幅150mm,厚さ50mmの直方体で,後に述べる送受信位置のとり方は,花崗岩供試体の場合と同様である.
\begin{figure}
\begin{center}
\includegraphics[clip,scale=0.5]{Figs/samples.eps}
\caption{
	超音波計測に用いた花崗岩供試体.
}
\label{fig:fig1}
\end{center}
	\vspace{-5mm}
\end{figure}
\subsection{超音波探触子(送信子)}
超音波の送信は,供試体表面に接触させた圧電超音波探触子で行う.
実験に用いた超音波探触子の外観は図-\ref{fig:fig2}のようであり,この図には
圧電素子を収納した筐体部分と,圧電素子がマウントされたウェッジ(シュー)部分が示されている.
内部に収納された圧電素子は,曲率半径が26.1mm,投影面積が幅25mm×長さ40mmの瓦状のもので,共振周波数は2MHzである.
圧電素子は,曲率半径を合わせて作成されたウェッジ上縁部に接着されており,
圧電素子で励起した縦波がウェッジ内部を伝播して先端部に集束するよう設計されている.
従ってウェッジ先端部を供試体に接触させて用いることで,供試体内部を円筒状に広がる弾性波が,
線状の接触部から励起される.なお,供試体に接触させるウェッジの先端部の幅と長さは1×50mmである.
%
高い超音波の集束効率を得るためには,センサーの開口を大きくとる必要がある.
また,ウェッジ内での減衰を抑えるためには,ウェッジの高さは低い方がよい.
一方で、入射点近傍での受信や,実験装置への取り付け,ウェッジ内部における多重反射波の抑制の
面からはこの逆のことが言える.上に述べた,圧電素子の曲率と幅はこれらの兼ね合いを考慮して
決定したものである.なお,圧電素子の長さは,表面波の波長をおよそ3mm程度と想定し,
その10倍以上となることを目安とした.
%
このような線集束型の探触子を用いることで,入射点と伝播方向が明確に設定される.
また,供試体表面から強い半円筒波状の超音波を励起することで,点波源から半球状の球面波を励起した場合に比べ,
幾何減衰の影響も小さくすることができ,信号/雑音比の点で有利になる.
%
さらに重要なことは伝播方向と伝播距離が共通する複数の地点で波形観測できる点にある.
花崗岩は音響異方性をもつ高減衰な材料である.そのため,伝播速度のゆらぎに関する議論では,方向と距離を指定して統計を取る必要があり,円筒波を励起する今回の探触子は,このことに配慮して選択したものである.
\begin{figure}[h]
\begin{center}
\includegraphics[clip,scale=0.9]{Figs/fig3.eps}
\caption{
	超音波の送信に用いた線集束探触子の外観.(a)正面,(b)側面から見た様子.
}
\label{fig:fig2}
\end{center}
	\vspace{-5mm}
\end{figure}
\subsection{超音波計測装置の構成}
実験に用いた超音波計測装置の構成を図-\ref{fig:fig3}に示す.計測装置は,3軸ステージ,レーザードップラー振動計(LDV),オシロスコープ,および高周波スクウェア−ウェーブパルサーで構成されている.
供試体は水平2軸,回転1軸の3軸ステージ上に固定し ,LDVによるレーザー照射位置を精確に調整する.
その際,送信探触子は,試験片表面に接触させて固定し,供試体とともに移動させる.
探触子の駆動はスクウェア−ウェーブパルサーを用いて行い,400Vの矩形パルス電圧を印加する.
受信にはLDVを用い,受信波形をオシロスコープへ転送し,4,096回の平均化を行った後,デジタル波形としてPCに収録する.サンプリング周波数は15MHz,計測時間範囲は200$\mu$秒とし,全ての計測は同じ条件で行った.
送信探触子の公称周波数は2MHzであるが,サンプリング周波数はやや低めに設定されている.
しかしながら,花崗岩供試体では低い周波数成分が主として透過し,多重散乱により振動の継続時間も送信パルス幅より長くなる傾向にある.このことに配慮し,ここではサンプリングレートを若干低めにし,計測時間範囲を余裕をもって設定することとした.
\begin{figure}[t]
\begin{center}
\includegraphics[clip,scale=0.45]{Figs/ut_setup.eps}
\caption{ 超音波計測装置の構成. }
\label{fig:fig3}
\end{center}
	\vspace{-15mm}
\end{figure}
\subsection{送受信位置}
図-\ref{fig:fig4}に,送信および受信領域の配置を示す.
ここで,${\cal S}$は送信位置,すなわち,線集束探触子のウェッジ先端が接触する位置を表し,この部分で試験片に鉛直動が加えられる.
${\cal R}$はLDVでスキャンする波形観測領域を表し,その大きさと形状は20mm$\times$30mmの矩形領域になっている.計測ピッチは$x$方向,$y$方向とも0.5mmとし,$\cal R$上の正方格子状に配置された観測点で計41×61=2,501点の超音波時刻歴波形を取得する.
なお,送信位置と受信領域の距離は20mmとしている.
これは,送信探触子の筐体に遮蔽され,レーザー光を直接照射することのできない領域が存在するためである.
アルミニウム供試体における観測では,座標原点を供試体表面の中央に取る他は,花崗岩供試体の場合に同じとした.

ここで,観測点格子の$x$および$y$軸方向間隔を,それぞれ,$\Delta x,\Delta y$とすれば,
$x$方向に$i$番目,$y$方向に$j$番目の観測点座標$(x_i,\, y_j)$は
\begin{equation}
	(x_i,\, y_j)=(x_0+i\Delta x,\, y_0+j\Delta y)
	\label{eqn:x_ij}
\end{equation}
と書くことができる.また,観測点が成す格子全体を${\cal G}$とすれば,
\begin{equation}
	{\cal G} = \left\{ 
	(x_i,\, y_j)\left| i=0,\dots N_x, \, j=0,\dots N_y  \right.
	\right\}
	\label{eqn:Grid}
\end{equation}
と表される.ただし,$N_x$と$N_y$は$x$および$y$軸方向の観測点数を表す.
実際の格子(観測)点数や格子間隔は,既に述べた通りであり,それらをまとめて示すと以下の通りとなる.
\begin{equation}
	\Delta x=\Delta y=0.5 {\rm mm}
	\label{eqn:grid_prms}
\end{equation}
\begin{equation}
	N_x=41, \, N_y=61
	\label{eqn:grid_nums}
\end{equation}
\begin{equation}
	(x_0,y_0)=(0,-15){\rm mm}
	\label{eqn:grid_corner}
\end{equation}
以下では,$t$を時間変数とし,位置$(x,y)$において観測した時刻歴波形を$a(x,y,t)$と表す.
簡単のため,$x,y$および$t$はいずれも連続変数として表記するが,$a(x,y,t)$に関する微分や積分などの演算を観測データに施す場合,観測点位置での値を使い,適宜離散化して評価する.
\begin{figure}[t]
\begin{center}
\includegraphics[clip,scale=0.4]{Figs/cod.eps}
\caption{
	超音波の送信位置$\cal S$と受信領域$\cal R$の配置.
}
\label{fig:fig4}
\end{center}
	\vspace{-10mm}
\end{figure}

\section{計測結果}
	%%%%%%%%%%%%%%%%%%%%%%%%%%%%%%%%%%%%%%%%%%%%%%%%%%%%%%%%%%%%%%%%%%%%%%%
計測によって得られた透過波波形の一例を図-\ref{fig:fig6}(a)に示す.
これは石英を透過した超音波の波形を横軸を時間,縦軸を速度振幅として
示したものである.一方,図\ref{fig:fig6}-(b)は,送信探触子の
シュー先端を,供試体に接触させず空気中で自由に振動させたときの
振動速度をLDVで計測した結果で,位相速度の算出において参照波形として
用いる.これら2つの波形を比較すると,(a)に示した透過波のうち,
概ね13[$\mu$sec]までの部分が直達波成分で,それ以後の振動は
板厚方向に往復を繰り返す多重反射波によるものであると判断される.
参照波形と透過波形では,透過経路長(供試体の板厚)が短く
到達時刻の差は1$\mu$secに満たない程度である.しかしながら,
鉱物種や観測点位置によって,到達時間には変動が生じている.
図-\ref{fig:fig7}$\sim$図-\ref{fig:fig9}はこのことを見るために,
全ての観測波形を鉱物種毎に示したものである.これらはいずれも,
横軸を時間,縦軸を波形番号(観測点番号)とし,速度振幅を
カラー表示したものである.
はじめに石英に対する結果である図\ref{fig:fig7}をみると,
大きな振幅をもつ初動部分の背後に,多重反射波が続く様子が
ほぼ全ての観測波形で見られる.また,青で示された速度振幅が負の
ピークをとる時間は,細かく振動している.これは到達時刻が一様でなく,
同じ石英でも位相速度が観測位置によって異なることを意味する.
このような到達時間差は,結晶異方性やマイクロクラックに起因
したもので,局所的な超音波の伝播速度を見ることで,結晶粒スケール
での非均質性や異方性を捉えることができることを示している.
以上の特徴は,図−\ref{fig:fig8}と図-\ref{fig:fig9}に示した
ナトリウム長石およびカリ長石にも概ね当てはまる.
一方で,これら二種類の長石に対する結果では,石英の場合に比べて
速度振幅の大きさが全体的に小さい.さらに,多重反射波の継続時間が短く,
特にカリ長石では直達波の背後に位相の揃った反射波成分がほとんど見られない.
また,振幅が負のピークを取る時刻の変動もカリ長石では他の鉱物種に比べて大きい.
ナトリウム長石とカリ長石は両者とも単斜晶系で,石英に比べて結晶構造の対称性が低い.
また,万成花崗岩のカリ長石にはへき開が認められる.
これらのことから,2種類の長石でピーク時刻の変動が石英に比べて大きいことには
異方性のタイプが関係し,カリ長石の多重反射波が弱いことはへき開による減衰が関係
していることが示唆される.
%このように,花崗岩中の弾性波は造岩鉱物の異方性やへき開による影響を受け,
%鉱物種毎に異なる挙動を示す.このことは,例えば風化によって鉱物の性状が変化しとき,
%弾性波伝播への影響が,風化の程度だけでなく鉱物組成にも依存したものとなる
%ことが期待でき,工学的に有用な変化がとなることを意味する。
%%弾性波から鉱物毎の特徴を逆解析できる可能性
\begin{figure}
\begin{center}
	\includegraphics[clip,scale=0.4]{Figs/Ascans.eps}
\caption{
	LDVので観測した時刻歴波形.(a)透過波波形(石英),(b)圧電探触子シュー先端部の振動波形(参照波形).
}
\label{fig:fig5}
\end{center}
%	\vspace{-5mm}
\end{figure}
\begin{figure}
\begin{center}
	\includegraphics[clip,scale=0.5]{Figs/Bscan_Q.eps}
	\caption{石英を透過した超音波の波形.}
	\label{fig:fig6}
\end{center}
%	\vspace{-10mm}
\end{figure}
\begin{figure}
\begin{center}
	\includegraphics[clip,scale=0.5]{Figs/Bscan_N.eps}
	\caption{ナトリウム長石を透過した超音波の波形.}
	\label{fig:fig7}
\end{center}
%	\vspace{-10mm}
\end{figure}
\begin{figure}
\begin{center}
	\includegraphics[clip,scale=0.5]{Figs/Bscan_K.eps}
	\caption{カリ長石を透過した超音波の波形.}
	\label{fig:fig8}
\end{center}
%	\vspace{-10mm}
\end{figure}

\section{位相速度}
	本節では,縦波位相速度$c_p$の計算方法をはじめに述べる.
次に,鉱物種ごとに求めた位相速度の平均と分布を求め,
鉱物種による位相速度の違いと速度分散の有無について明らかにする.
\subsection{位相速度の計算方法}
位相速度$c_p$を,図\ref{fig:fig5}-(b)に示した参照波形と個々の透過波形の
位相差から次の手順で求める.以下では,計測した透過波の時間波形を$a_{raw}(t)$,
参照波形を$a_{ref}(t)$と表す.
\begin{enumerate}
\item
直達波成分の切り出し: 
位相速度の計算には直達波を用いるため,観測波形$a_{raw}(t)$に窓関数$W(t;T_w)$を作用させ
\begin{equation}
	a(t):=W(t-t_b;\tau)a_{raw}(t)
\end{equation}
とすることで,直達波$a(t)$を切り出す.ここで,$t_b$は時間軸上における窓関数の位置を,
$\tau$は窓関数の幅を決定するパラメータを表す.本研究では,$W(t;\tau)$として
次の式で与えられるHann Windowを用いる.
\begin{equation}
	W(t;\tau):=\left\{
	\begin{array}{cl}
		\frac{1}{2}\left[ 1-\cos \left\{ \pi\left(1+t/\tau \right) \right\} \right],
		 & \left| t \right| < \tau \\
		0, & {\rm otherwise}
	\end{array}
	\right.
\end{equation}
Hann Windowのパラメータ$\tau$は$1MHz$以上の周波数成分を捉えることができるよう
$\tau=0.6$[$\mu$sec]とし,位置$t_b$は直達波の位置を振幅の正負が切り替わる
時刻として見積もって与えるようにした.
なお,Hann Windowを選択した理由は,少数のパラメータを設定するだけでよく使い易いこと,
有限な台をもつため,多重反射波の成分を完全に切り落とすことができることの二点である.
\item
フーリエ変換と逆畳み込み:
時間に関するフーリエ変換を,角周波数を$\omega$として
	\begin{equation}
		F(\omega) ={\cal F} \left\{ f(t)\right\}:=\int f(t) e^{i\omega t}dt 
		\label{eqn:Fourier}
	\end{equation}
で定義し,$a(t)$と$a_{ref}(t)$のフーリエ変換
	\begin{equation}
		A(\omega):={\cal F} \left\{ a(t) \right\}, \ \ 
		A_{ref}(\omega):={\cal F} \left\{ a_{ref} (t)\right\}
	\end{equation}
をFFTでそれぞれ求める.
これらの逆畳み込みをとることで,信号$a(t)$と$a_{ref}(t)$の位相差$\Delta \phi(\omega)$
を次の式から求める.
\begin{equation}
	\frac{A(\omega)}{A_{ref}(\omega)}
	=
	\left|
	\frac{A(\omega)}{A_{ref}(\omega)}
	\right|
	e^{-i\Delta \phi(\omega)}
	\label{eqn:deconv}
\end{equation}
\item
位相差のアンラップ:
	位相差$\Delta \phi(\omega)$を低周波側から高周波側に向かってアンラップする.
	このとき,周波数$0$の近傍ではノイズの影響が大きく誤った位相の値を持つ可能性が高いため,
	参照波の周波数帯域下限付近である500[kHz]を起点としてアンラップを行う.
	なお,周波数500[kHz]であれば,今回の実験条件において縦波の波長が供試体の
	板厚(透過距離)を下回ることは無く,位相の不定性は問題にならない.
\item
位相速度の計算:
	アンラップされた位相差$\Delta \phi(\omega)$と,供試体の厚さ$h$(透過距離)を用い,
\begin{equation}
	c_p(\omega)=\frac{\omega h}{\Delta \phi(\omega)}
	\label{eqn:cp_phi}
\end{equation}
	で位相速度を周波数毎に計算する.
\end{enumerate}
以上の手順で求めた位相速度と,対応する鉱物種を明示する場合には,
位相速度を$c_p^{(\alpha)}$と書くことにする.ここに,$\alpha$はQt, Na, Kのいずれかで,
それぞれ,石英,ナトリウム長石,カリ長石を意味する.また,位相速度の値は計測波形の数$N_\alpha$
だけ求められる.そこで,第$m$番目の計測波形に対して得られた位相速度を指す場合は
$c_{p,m}^{(\alpha)}$と表記し,鉱物種$\alpha$に対する波形データについて
得られた位相速度の全体を指す場合には$\left\{ c_{p,m}^{(\alpha)}\right\}$と書く.
さらに,鉱物種$\alpha$に対する位相速度の平均を
\begin{equation}
	\left< c_p^{(\alpha)}\right>:= \frac{1}{N_\alpha}\sum_{m}^{N_\alpha} c_{p,m}^{\alpha }
	\label{eqn:}
\end{equation}
と表す.以下では,この意味での位相速度の平均や分布について議論する.
\subsection{位相速度の平均}
鉱物種毎に求めた位相速度の平均と周波数の関係を図\ref{fig:figfig9}に示す.
この図は,周波数に対して$\left< c_p^{(Q)}\right>, 
\left< c_p^{(Na)}\right>$,および $\left< c_p^{(K)}\right>$をプロット
したもので.平均的な速度分散関係を示したものである.
この結果から明らかなように,鉱物種によらず位相速度はほぼ一定値となっており,
ここに示した周波数帯域ではほとんど速度分散は見られないことが分かる.
また,位相速度の大きさは
\begin{equation}
	\left< c_p^{(Q)} \right> 
	< 
	\left< c_p^{(K)} \right> 
	< 
	\left< c_p^{(Na)}\right>
	\label{eqn:}
\end{equation}
で周波数によって大小関係が逆転することは無い.
ただし,いずれの鉱物種でも位相速度は概ね5.4[km/s]
で,鉱物種間での差も約$\pm$ 0.2[km/s],すなわち3.7$\%$程度と
平均値を見る限りあまり大きくない.
\begin{figure}
\begin{center}
	\includegraphics[clip,scale=0.5]{Figs/cp_mean_f.eps}
	\caption{位相速度の平均値と周波数の関係.}
	\label{fig:fig9}
\end{center}
	\vspace{-10mm}
\end{figure}
\begin{figure}
\begin{center}
	\includegraphics[clip,scale=0.5]{Figs/hist_Q.eps}
	\caption{位相速度の分布(石英).}
	\label{fig:fig10}
\end{center}
	\vspace{-10mm}
\end{figure}
\begin{figure}
\begin{center}
	\includegraphics[clip,scale=0.5]{Figs/hist_N.eps}
	\caption{位相速度の分布(ナトリウム長石).}
	\label{fig:fig11}
\end{center}
	\vspace{-10mm}
\end{figure}
\begin{figure}
\begin{center}
	\includegraphics[clip,scale=0.5]{Figs/hist_K.eps}
	\caption{位相速度の分布(カリ長石).}
	\label{fig:fig12}
\end{center}
	\vspace{-10mm}
\end{figure}
%%%%%%%%%%%%%%%%%%%%%%%%%%

\section{非均質媒体モデルによる波動伝播解析}
	最後に,実験で得られた鉱物種毎の位相速度分布を用いて行った波動伝播解析の結果を示す.
簡単のため位相速度のコントラストによって鉱物粒を表現した非均質等方性体
によって花崗岩を模擬し,2次元スカラー波の伝播散乱解析を行う.
位相速度の空間分布は、鉱物種をごとに区別して設定した場合と、
鉱物種による区別を行うことなく所定のガウス分布に従うものとして与えた
2種類の場合について行う.両者の結果を比較することにより,
鉱物種毎に位相速度を設定することの意義を示す.
\subsection{問題設定}
図\ref{fig:fig15}-(a)に示す矩形領域$D$における波動伝播問題を考える.
領域$D$における位相速度$c_p(\fat{x})$と密度$\rho(\fat{x})$は,
予め与えられているものとする.
ただし,鉱物種による密度の差は小さいことから,以下では
$\rho$は場所に依らず一定として扱う.
この領域の左側の境界($x=0$)に一様な応力$\sigma_{in}(t)$を加え,$x>0$の方向に伝播する
平面波を励起する.境界に加える応力の時間変化は図\ref{fig:fig15}-(b)に示す,
周波数1MHzのコサイン波をガウス関数で振幅変調した波形を用いる.
$y$方向には十分長い領域を想定し,$D$の上下の境界$y=0,H$は周期境界条件
を課す.一方,右側の境界$x=W$では応力はゼロであるとして,このとき
$D$内を各所で散乱を起こしながら伝播する平面波の挙動を調べる.
媒体は位相速度が場所によって異なるため,厳密な意味での平面波は存在し得ない.
しかしながら,以下で見るように
位相の揃った初動部分は概ね1次元的な$x>0$方向への進行波と見ることができるため,
ここでは便宜上平面波の励起や伝播という言い方をする。
\begin{figure}
\begin{center}
	\includegraphics[clip,scale=0.5]{Figs/model.eps}
	\caption{数値解析モデルと差分格子}
	\label{fig:fig15}
\end{center}
\end{figure}
\subsection{波動伝播問題の定式化と数値解析法}
スカラー波動場の支配方程式は,等方線形弾性体の運動方程式において
せん断剛性$\mu$をゼロとすることによって得られ,次のように表すことができる.
\begin{equation}
	\rho \dot{\fat{v}}=\nabla \sigma
	\label{eqn:eq_mot}
\end{equation}
\begin{equation}
	\dot{\sigma}=\lambda \nabla \cdot \fat{v}
	\label{eqn:Hooke}
\end{equation}
ここに,$\fat{v}$は速度ベクトルを,$\sigma$は応力テンソルの球成分を,
$\lambda$はラメ定数を表す.位相速度とラメ定数の関係は
$\lambda =\rho c_p^2$で与えられる.
媒体は時刻$t=0$において静止状態にあるものとし,初期条件を
\begin{equation}
	\sigma(\fat{x},0)=0, \ \ \fat{v}(\fat{x},0)=\fat{0}, \ \ (\fat{x} \in D)
	\label{eqn:IC}
\end{equation}
で与える.境界条件は上に述べて問題設定から
次のように書くことができる.
\begin{equation}
	\sigma(0,y,t)=\sigma_{in}(t), \ \ (t>0)
	\label{eqn:}
\end{equation}
\begin{equation}
	\sigma(W,y,t)=0, \ \ (t>0)
	\label{eqn:}
\end{equation}
\begin{equation}
	\sigma(x,0,t)=\sigma(x,H,t)  \ \ (x=W, t>0)
	\label{eqn:}
\end{equation}
以上で表される初期値-境界値問題を解くことで,領域$D$における平面波の伝播,散乱挙動を
調べることができる.数値解析には,有限差分法を用いる.
差分法解析を行うための格子配置を図\ref{fig:fig15}-(c)に示す.
この図は,差分格子の単位セルを示したもので,丸が速度ベクトル$\fat{v}$,
ひし形が応力$\sigma$の計算格子を表す.
非均質媒体を扱うためにはこれらの格子点に加え,密度$\rho$と位相速度$c_p$を与える
格子を決めておく必要がある.ここでは,位相速度は応力の計算格子と,
密度は速度の計算格子で与える.このようにすることで,位相速度の不連続部を
平均化すること無く取り扱うことができる.
密度に関しては一定と近似しているため,どの格子点に密度評価点を対応付けても違いはない.
式(\ref{eqn:eq_mot})と(\ref{eqn:Hooke})の空間微分は,図\ref{fig:fig15}-(c)の格子上で
中央差分で近似する.一方,時間の微分はリープフロッグ法を用いて離散化し,
速度$\fat{v}$と応力$\sigma$を交互に陽解法で求める.
\subsection{解析条件}
領域$D$のサイズを
\begin{equation}
	W=50[{\rm mm}], \ \  H=30[{\rm mm}]
	\label{eqn:dom_size}
\end{equation}
とし,時刻$ 0 <t < 15[\mu$sec]における解$\fat{v}$と$\sigma$を差分法によって求めた.
位相速度の空間分布は,実験で得た頻度分布(図\ref{fig:fig13})から鉱物種毎に
サンプリングし,鉱物粒内で一定の値を与えた.このようにして設定した位相速度の
空間分布の一例を図\ref{fig:fig16}-(a)に示す.
この図は領域$D$における位相速度$c_p(\fat{x})$の値を示したもので,
鉱物粒の分布は以下のようにして与えている.
はじめに,花崗岩供試体のスキャナ画像から,解析モデルに用いいる範囲を切り出し,
画像ピクセル毎に鉱物種を特定する.
次に,鉱物種ごとにピクセル間のユークリッド距離に基づいてクラスタリングを行い,
同一のクラスターに属するピクセルを一つの鉱物粒として扱う.
クラスタリングは鉱物種毎に行うことで,石英,ナトリウム長石,カリ長石の結晶粒
を定義することができる.なお,黒雲母の位相速度データは得られていないため,
黒雲母と判定させれたピクセルは最近傍のクラスターに振り分け,そのクラスターの
鉱物種で代替する.この方法ではクラスター数を調整することで,結晶粒の数を
任意に調整できる。その結果,平均的な結晶粒のサイズもコントロールできる。
さらに,クラスターの初期位置を乱数を使って設定することで,多数の
ランダム媒体を簡単に生成できる.ここでは,クラスターの数は,それぞれの鉱物種
で200個,合計600の鉱物粒が生成されている.その結果,結晶粒の粒径はおよそ1.6mm程度
となっている.なお,花崗岩供試体の画像上で特定した、鉱物種は黒雲母以外は事後的に
変更しないため,クラスタリングによって鉱物分布自体が変わることはない.
差分法解析を行う際の格子間隔$\Delta x, \Delta y$は,位相速度マップの
作成に用いた画像の1ピクセルに相当するサイズとした.
一方時間ステップ$\Delta t$は,位相速度の最大値$c_{max}$に対して,
安定条件を満足するように決定した.具体的には,これらのステップ長は
以下のようである.
\begin{equation}
	\Delta x = \Delta y= 0.055[{\rm mm}], \ \ \Delta t=0.04[\mu{\rm sec}]
	\label{eqn:}
\end{equation}
\subsection{解析結果}
\begin{figure}
\begin{center}
	\includegraphics[clip,scale=0.5]{Figs/field.eps}
	\caption{(a)数値解析モデルに設定した位相速度.(b),(c)速度場のスナップショット.}
	\label{fig:fig16}
\end{center}
\end{figure}

\begin{figure}
\begin{center}
	\includegraphics[clip,scale=0.5]{Figs/decay.eps}
	\caption{振幅比の伝播距離による変化.}
	\label{fig:fig17}
\end{center}
\end{figure}

\section{まとめ}
本研究では,花崗岩の鉱物粒スケールのおける縦波位相速度を,局所的な超音波計測よって推定した.
その結果,実験に用いた花崗岩供試体の主要造岩鉱物である,石英,ナトリウム長石,カリ長石の
それぞれについて,位相速度の頻度分布を得た.これらの頻度分布の平均値は鉱物種による差が小さく,
いずれも5.4[km/s]程度と典型的な花崗岩の位相速度に近い値をとり,周波数1MHzから2MHz程度の帯域
ではほとんど速度分散も無いことが分かった.一方,頻度分布の形状や分布幅は鉱物種によって大きく異なり,
次のような特徴があることが明らかとなった.
\begin{itemize}
\item
石英の位相速度は,平均値周辺に集中した対称な頻度分布を持つ.
\item
カリ長石の位相速度は,標準偏差が0.9[km/s]と大きく非対称な頻度分布を持つ. 
\item
ナトリウム長石の位相速度は,複数のピークをもつ頻度分布を描き,分布幅は
石英とカリ長石の中間程度の値となる.
\end{itemize}
以上の結果を,文献に与えられた単結晶鉱物の弾性係数から計算した位相速度と比較する
ことで,花崗岩中の鉱物粒はマイクロクラックの影響により,位相速度がき裂の無い単結晶
に比べて明らかに小さな値を取ることを確認した.一方,分布幅に関して言えば,
単結晶に比べて花崗岩中の石英は頻度分布の分散が小さく,き裂の存在が異方性を
弱める方向に寄与している可能性があることが分かった.さらに,カリ長石は,単結晶
同様,位相速度の頻度分布幅が広く,き裂の存在下でも強い異方性を示すことを明らか
とすることができた.最後に,これら鉱物種に応じた位相速度の頻度分布を考慮して
波動伝播解析を行ったところ,位相速度が単一のガウス分布に従うと仮定した
場合とは異なる散乱減衰挙動が見られることが分かった.
このことは,花崗岩のような複数の鉱物から構成される非均質媒体を伝わる弾性波の
挙動を詳しく調べるためには,構成鉱物毎に位相速度あるいは弾性係数が従う確率分布を
与えた解析が必要となることを示していると考えられる.
%逆に言えば,花崗岩中を伝播する弾性波
%は鉱物組成に反応すると言え,両者の詳しい関係を知ることができれば,弾性波計測結果から
%粒径分布やき裂の量を推定する非破壊検査法として利用できる可能性がある.
%こういった利用方法を
今後は,供試体の劣化による位相速度分布の変化を実験的に調べることや,
横波についても同様な計測と位相速度分布の推定を行うことが課題となる.
また,波動伝播挙動のどのような側面に位相速度の確率分布の特徴が反映されるかを,
数値シミュレーションによって特定することも重要な課題の一つとなる.
これらを両輪として研究を行うことが,岩石を始めとする複雑なランダム不均質媒体の弾性波試験を
より定量的かつ信頼性の高いものにしていくにあたって重要と考えられ,
本研究はそのような取り組みの一環として位置づけられるものと言える.
%%%%%%%%%%%%%%%%%%%%%%%%%%%%%%%%%%%%%%%%%%%%%%%%%%%%%%%%%%%%%%%%%%%%%%%%%%%%%
\\

{\gt 謝辞:}
本実験に用いた花崗岩供試体は浮田石材店代表浮田隆司氏に提供頂いた.
また本研究の推進には,科学研究費補助金(基盤研究©課題番号\#18K04334)の補助を受けた.
併せて謝意を表す.
%%%%%%%%%%%%%%%%%%%%%%%%%%%%%%%%%%%%%%%%%%%%%%%%%%%%%%%%%%%%%%%%%%%%%%%%%%%%%
%\newpage
%\lastpagecontrol[2cm]{13.7cm}
\begin{thebibliography}{99}
\begin{spacing}{1.175}
\bibitem{Kudo1}
	工藤洋三, 橋本堅一, 佐野修, 中川浩二: 花崗岩の力学的異方性と岩石組織欠陥の分布,
	土木学会論文集, 第370号/III-5, pp.189-197, 1986.
\bibitem{Kudo1991}
	工藤洋三, 橋本堅一, 佐野修, 中川浩二: 花崗岩内に発生するクラックと鉱物粒の関係,
	資源・素材学会誌, 第107巻,第7号, pp.423-427, 1991.
\bibitem{Chin}
	陳友晴,西山孝,喜多浩之,佐藤稔紀: 微小クラックの分類による稲田花崗岩と栗橋花崗閃緑岩の力学的弱面について,
	応用地質,第38巻,第4号,pp.196-204,1997.
\bibitem{Takagi}
	高木秀雄, 三輪成徳, 横溝佳侑, 西嶋圭, 円城寺守, 水野崇, 天野健治:土岐花崗岩中の石英に発達するマイクロクラックの三次元
	方位分布による古応力場の復元と生成環境, 地質学雑誌, 第114巻, 第7号, pp.321-335, 2008.
\bibitem{Griffiths}
	Griffiths, L., Lengline, O., Heap, M.,J., Baud, P., and Schmittbuhl, J.: Thermal cracking in Westerly granite monitored
	using direct wave velocity, coda wave interferometry and acoustic emissions, {\it JGR Solid Earth}, Vol.123, No.3, pp.2246-2261, 2018.
\bibitem{Sato}
	Sato, H., Fehler, M.C., and Maeda, T.:Seismic wave propagation and scattering in the heterogeneous earth, 
	Springer, 2012.
\bibitem{FEM}
	Pamel, A., V., Sha, G., Rokhlin, S., I., and Lowe, M., J., S.:
	Finite-element modelling of elastic wave propagation and scattering within heterogeneous media, 
	{\it Proc. R. Soc. A}, 473:20160738.
\bibitem{AGU}
	Anderson, O., L., Schreiber, E., Liebermann, R., C., and Soga, H.: 
	Some elastic constant data on minerals relevant to geophysics, 
	{\it Review of Geophysics}, Vol.4, No.4, pp.491-524, 1968.
\bibitem{Kawamura}
	河村雄行: 分子シミュレーション鉱物学のすすめ,
	鉱物学雑誌,第28巻,第4号,pp.151-158,1999.
%\bibitem{Bass}
%	Bass, J., D.: Elasticity of minerals, glasses, and melts, 
%	{\im Mineral Physics and Crystallography:A handbook of physical constants }, Vol.2,pp.45-63,  
\bibitem{Pamel}
	Pamel, A., V., Sha, G., Rokhlin, S.,I., and Lowe, M.,J.,S.: 
	Finite-element modelling of elastic wave propagation and scattering within heterogeneous media, 
	{\it Proc. R. Soc. A}, 473:2016738, 2016.
\bibitem{Kudo2}
	工藤洋三, 橋本堅一, 佐野修, 中川浩二:瀬戸内地方の採石場における花崗岩石の異方性, 
	土木学会論文集, 第382号/III-7, pp.45-53, 1987.
\newpage
\bibitem{Sano1}
	佐野修, 工藤洋三, 河嶋智, 水田義明:異方性体としての花崗岩の弾性率に関する実験的研究, 
	材料, 第37巻, 第418号, pp.84-90, 1987.
%\bibitem{Sano2}
%	佐野修, 民部雅史, 平野亮, 工藤洋三, 水田義明:弾性的対称性未知の岩石の弾性定数決定に関する研究, 
%	材料, 第40巻, 第449号,pp.96-102, 1990.
%\bibitem{Nishizawa1996}
%	西澤修, 雷興林, 佐藤隆司:不均質媒体での地震波伝モデル実験-レーザードップラー速度計を用いた波動計測-
%	, 地震調査所月報, 第47巻, 第4号, pp.209-222, 1996.
\bibitem{Sivaji}
	Sivaji, C., Nishizawa, O., Kitagawa, G., and Fukushima, Y.:A physical-model study of the statistics of seismic waveform fluctuation in random heterogeneous media, 
	{\it Geophys. J. Int.}, Vol.148, pp.575-595, 2002. 
\bibitem{Fukushima}
	Fukushima, Y., Nishizawa, O., Sato, H., and Ohtake, M.:Laboratory study on scattering characteristics of shear waves 
	in rock samples, {\it Bulltine of Seismological Society of America}, Vol.93, No.1, pp.253-263, 2003.
\bibitem{Muller}
	Muller, G., Roth, M. and Korn, M.:Seismic-wave traveltimes in random media,
	{\it Geophys. J. Int.}, Vol.110, pp.29-41, 1992. 
\bibitem{Korn}
	Korn, M.:Seismic waves in random media, 
	{\it Journal of Applied Geophysics}, Vol.29, pp.247-269, 1993.
\bibitem{Spetzler2001}
	Spetzler, J. and Snieder, R.:The effect of small-scale heterogeneity on the arrival time of waves, 
	{\it Geophys. J. Int.}, Vol.145, pp.786-796, 2001. 
\bibitem{Spetzler}
	Spetzler, J., Sivaji, C., Nishizawa, O., and Fukushima, Y.:A test of ray theory and scattering theory based on
	a laboratory experiment using ultrasonic waves and numerical simulation by finite-difference method, 
	{\it Geophys. J. Int.}, Vol.148, pp.165-178, 2002. 
\bibitem{Baig}
	Baig, A., M., and Dahlen, F., A.: Traveltime biases in random media and S-wave discrepancy, 
	{\it Geophys. J. Int.}, Vol.1158 pp.992-938, 2004. 
\bibitem{Kimoto}
	木本和志,岡野蒼,斎藤隆泰,佐藤忠信,松井裕哉: 超音波計測に基づく花崗岩中の表面波伝播特性に関する研究,
	土木学会応用力学論文集A2(応用力学),第76巻,第2号,pp.I\_97-I\_108, 2020.
%\bibitem{RockPhys}
%	ゲガーン, Y., パルシアウスカス, V.: 
%	岩石物性入門, シュプリンガー・ジャパン, 2008. 
%\bibitem{NishizawaI}
%	西澤修:岩石中の地震波伝播I:不均質媒体のモデル化と弾性波速度, 地学雑誌, 第114巻, 第6号,  pp.921-948,  2005.
%\bibitem{Nishizawa2001}
%	西澤修, 雷興林, チャダラム シバジ:不均質媒質での地震波伝播モデル実験, 
%	地震, 第54巻, pp.171-183, 2001.
%\bibitem{Okubo2012}
%	大久保 慎人, 雑賀 敦, 鈴木 貞臣, 中島 唯貴: 地震動観測による地震波速度と岩石物性試験による弾性波速度の関係
%	-段発発破波形の相関による地震波速度構造推定-: 地震, 第65巻, pp.21-30, 2012.
%\bibitem{Rwk_textbook}
%	Klaffter, J., and Sololov,I.M.,秋元 琢磨(訳): ランダムウォークはじめの一歩,共立出版, 2018.
\end{spacing}
\end{thebibliography}
\begin{flushright}
	\small
	\bf{ (Received June 18, 2021)\\
	(Accepted November **, 2021)}
\end{flushright}
\newpage
\lastpagecontrol[0.0cm]{18.0cm}
\end{document}

%\lastpagesettings
%\begin{minipage}[c]{13.7cm}
%\end{minipage}
%\lastpagecontrol[0cm]{13.7cm}
%\begin{multicols}{1}
%-------------------------------------------------
%-------------------------------------------------
%\end{multicols}

